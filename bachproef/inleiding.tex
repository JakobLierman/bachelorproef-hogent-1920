%%=============================================================================
%% Inleiding
%%=============================================================================

\chapter{\IfLanguageName{dutch}{Inleiding}{Introduction}}
\label{ch:inleiding}

\epigraph{If the user can't use it, it doesn't work.}{\textit{Susan Dray}}

In digitale tijden als deze is de vraag naar nieuwe software groot. Programmeurs en IT-bedrijven hebben hun handen vol. Zowat elke sector wil mee zijn met de digitale boot. Je komt hedendaags overal software tegen; de computer op het werk, de smartphone in je broekzak, het bedieningspaneel van een grote kraan, de kassa in de supermarkt, $\dots$ Je gebruikt hoogstwaarschijnlijk tal van software in het dagelijkse leven. Maar hoe leer je nu het best omgaan met deze computerprogramma's werken? Programma's bevatten vaak heel veel functies. Veel van deze functies blijven echter onbenut omdat de gebruiker niet voldoende begrijpt hoe deze functies in zijn werking treden. De programmeurs achter deze software hebben vaak hun handen al vol met het ineen knutselen van al deze functionaliteit dat zij zelf geen tijd hebben om te kijken hoe de eindgebruiker met deze functionaliteit omspringt.

\epigraph{How do I explain what I do at a party? The short version is that I say I humanize technology.}{\textit{Fred Beecher, Director of UX, The Nerdery}}

Hier komt de UX-designer aan bod. Een UX-designer zorgt er voor dat software bruikbaar is voor de eindgebruiker. Taken die zijn job omschrijven omvatten, maar zijn niet beperkt tot maken van prototypes, testen van (deel)producten bij gebruikers, observeren van gedrag van gebruikers op bepaalde functionaliteiten en stukken software, \textit{user flows} creëren en ook onderzoek doen naar de doelgebruiker~\autocite{White2020}. De UX-designer zal dus ook een flow creëren dat ervoor zorgt dat de gebruiker alle functionaliteit van de applicatie goed begrijpt en snel onder de knie heeft.

Eén van de bekendste implementaties hiervan is de ``onboarding''. Je komt vaak in contact met onboarding wanneer je de applicatie voor de eerste maal opstart. Zo'n onboarding kan zeer verschillend zijn van applicatie tot applicatie.
Er bestaan uiteraard meer manieren om de gebruiker de weg te wijzen doorheen software. Een simpele \textit{tooltip} of zelfs een help-pagina of leerplatform doet ook wonderen.

\section{\IfLanguageName{dutch}{Probleemstelling}{Problem Statement}}
\label{sec:probleemstelling}

Uit je probleemstelling moet duidelijk zijn dat je onderzoek een meerwaarde heeft voor een concrete doelgroep. De doelgroep moet goed gedefinieerd en afgelijnd zijn. Doelgroepen als ``bedrijven,'' ``KMO's,'' systeembeheerders, enz.~zijn nog te vaag. Als je een lijstje kan maken van de personen/organisaties die een meerwaarde zullen vinden in deze bachelorproef (dit is eigenlijk je steekproefkader), dan is dat een indicatie dat de doelgroep goed gedefinieerd is. Dit kan een enkel bedrijf zijn of zelfs één persoon (je co-promotor/opdrachtgever).

\section{\IfLanguageName{dutch}{Onderzoeksvraag}{Research question}}
\label{sec:onderzoeksvraag}

Wees zo concreet mogelijk bij het formuleren van je onderzoeksvraag. Een onderzoeksvraag is trouwens iets waar nog niemand op dit moment een antwoord heeft (voor zover je kan nagaan). Het opzoeken van bestaande informatie (bv. ``welke tools bestaan er voor deze toepassing?'') is dus geen onderzoeksvraag. Je kan de onderzoeksvraag verder specifiëren in deelvragen. Bv.~als je onderzoek gaat over performantiemetingen, dan 

\section{\IfLanguageName{dutch}{Onderzoeksdoelstelling}{Research objective}}
\label{sec:onderzoeksdoelstelling}

Wat is het beoogde resultaat van je bachelorproef? Wat zijn de criteria voor succes? Beschrijf die zo concreet mogelijk. Gaat het bv. om een proof-of-concept, een prototype, een verslag met aanbevelingen, een vergelijkende studie, enz.

\section{\IfLanguageName{dutch}{Opzet van deze bachelorproef}{Structure of this bachelor thesis}}
\label{sec:opzet-bachelorproef}

% Het is gebruikelijk aan het einde van de inleiding een overzicht te
% geven van de opbouw van de rest van de tekst. Deze sectie bevat al een aanzet
% die je kan aanvullen/aanpassen in functie van je eigen tekst.

De rest van deze bachelorproef is als volgt opgebouwd:

In Hoofdstuk~\ref{ch:stand-van-zaken} wordt een overzicht gegeven van de stand van zaken binnen het onderzoeksdomein, op basis van een literatuurstudie.

In Hoofdstuk~\ref{ch:methodologie} wordt de methodologie toegelicht en worden de gebruikte onderzoekstechnieken besproken om een antwoord te kunnen formuleren op de onderzoeksvragen.

% TODO: Vul hier aan voor je eigen hoofstukken, één of twee zinnen per hoofdstuk

In Hoofdstuk~\ref{ch:conclusie}, tenslotte, wordt de conclusie gegeven en een antwoord geformuleerd op de onderzoeksvragen. Daarbij wordt ook een aanzet gegeven voor toekomstig onderzoek binnen dit domein.