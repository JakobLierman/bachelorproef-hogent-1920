%%=============================================================================
%% Inleiding
%%=============================================================================

\chapter{\IfLanguageName{dutch}{Inleiding}{Introduction}}
\label{ch:inleiding}

\epigraph{If the user can't use it, it doesn't work.}{\textit{Susan Dray}}

In digitale tijden als deze is de vraag naar nieuwe software groot. Programmeurs en IT-bedrijven hebben hun handen vol. Zowat elke sector wil mee zijn met de digitale boot. Je komt hedendaags overal software tegen; de computer op het werk, de smartphone in je broekzak, het bedieningspaneel van een grote kraan, de kassa in de supermarkt, $\dots$ Je gebruikt hoogstwaarschijnlijk tal van software in het dagelijkse leven. Maar hoe leer je nu het best omgaan met deze computerprogramma's? Programma's bevatten vaak heel veel functies. Veel van deze functies blijven echter onbenut omdat de gebruiker niet voldoende begrijpt hoe deze functies in zijn werking treden. De programmeurs achter deze software hebben vaak hun handen al vol met het ineen knutselen van al deze functionaliteit dat zij zelf geen tijd hebben om te kijken hoe de eindgebruiker met deze functionaliteit omspringt.

\epigraph{How do I explain what I do at a party? The short version is that I say I humanize technology.}{\textit{Fred Beecher, Director of UX, The Nerdery}}

Hier komt de UX-designer aan bod. Een UX-designer zorgt er voor dat software bruikbaar is voor de eindgebruiker. Taken die zijn job omschrijven omvatten, maar zijn niet beperkt tot maken van prototypes, testen van (deel)producten bij gebruikers, observeren van gedrag van gebruikers op bepaalde functionaliteiten en stukken software, \textit{user flows} creëren en ook onderzoek doen naar de doelgebruiker~\autocite{White2020}. De UX-designer zal dus ook een flow creëren dat ervoor zorgt dat de gebruiker alle functionaliteit van de applicatie goed begrijpt en snel onder de knie heeft.

Eén van de bekendste implementaties hiervan is de ``onboarding''. Je komt vaak in contact met onboarding wanneer je de applicatie voor de eerste maal opstart. Zo'n onboarding kan zeer verschillend zijn van applicatie tot applicatie.
Er bestaan uiteraard meer manieren om de gebruiker de weg te wijzen doorheen software. Een simpele \textit{tooltip} of zelfs een help-pagina of leerplatform doet ook wonderen.

\section{\IfLanguageName{dutch}{Probleemstelling}{Problem Statement}}
\label{sec:probleemstelling}

Bij veel software-ontwikkelaars en -designers stelt de vraag zich frequent als een bepaalde implementatie van onboarding of in-app user training wel het gewenste effect bekomt. Er bestaan verscheidene manieren om dit in een applicatie te verwerken, om te weten als de ene manier een beter resultaat boekt dan de andere is een onderzoek nodig.

We bekeken de voordelen van onboarding en in-app user training nu voornamelijk vanuit het oogpunt van de eindgebruiker. Echter kan dit ook effect hebben op andere aspecten binnen een bedrijf. Deze scriptie zal ook de effecten op het klantbehoud analyseren. Er zal dus onderzocht worden indien de klant zich meer geneigd gaat voelen een bepaalde applicatie met een andere en/of betere implementatie van de verschillende technieken frequenter te gebruiken.
% TODO - Staven met literatuur ???
% TODO - Laatste zin verwijst effectief naar onderzoeksvraag

\section{\IfLanguageName{dutch}{Onderzoeksvraag}{Research question}}
\label{sec:onderzoeksvraag}

\subsection{Hoofdonderzoeksvraag}
\label{sec:hoofdonderzoeksvraag}

Zoals reeds aangehaald in sectie~\ref{sec:probleemstelling} zal dit onderzoek zich focussen op implementaties van onboarding en in-app user training en de gevolgen van deze implementaties. Daaruit vloeit volgende hoofdonderzoeksvraag voort:

\begin{itemize}
    \item Kan een (betere) onboarding en in-app user training ervoor zorgen dat de eindgebruiker een beter inzicht heeft op de totale functionaliteit van een grote applicatie?
\end{itemize}

\subsection{Deelonderzoeksvragen}
\label{sec:deelonderzoeksvragen}

Ter ondersteuning van de hoofdonderzoeksvraag zijn er ook nog enkele deelonderzoeksvragen opgesteld:

\begin{itemize}
    \item Hoe een grote hoeveelheid aan functionaliteiten beheersbaar houden voor de eindgebruiker?
    \item Heeft (het gebrek aan) in-app user training effect op de gebruiksduur en/of levensduur van de applicatie?
    \item Hoe de eindgebruiker wegwijs maken in een grote applicatie?
\end{itemize}

Doorheen deze scriptie zal op deze deelonderzoeksvragen een antwoord geformuleerd worden.

\section{\IfLanguageName{dutch}{Onderzoeksdoelstelling}{Research objective}}
\label{sec:onderzoeksdoelstelling}

Het hoofddoel van dit onderzoek is het aantonen van de werking en gevolgen van verschillende technieken om de eindgebruiker familiair te maken met de applicatie. Door middel van een proof-of-concept zal worden aangetoond welke technieken het meest gewenste resultaat hebben op de eindgebruiker.

Een tweede doel bestaat uit het onderzoeken als (het gebrek aan) deze technieken ook effect heeft op het klantbehoud bij de applicatie.

Een laatste doel is om er voor te zorgen dat ontwikkelaars, die dit lezen, een idee krijgen van de UX-technieken die men kan gebruiken bij de implementatie van hun applicatie. Alsook UX-designers kunnen deze scriptie raadplegen bij het uitdenken van hun user flows.

\section{\IfLanguageName{dutch}{Opzet van deze bachelorproef}{Structure of this bachelor thesis}}
\label{sec:opzet-bachelorproef}

% Het is gebruikelijk aan het einde van de inleiding een overzicht te geven van de opbouw van de rest van de tekst. Deze sectie bevat al een aanzet die je kan aanvullen/aanpassen in functie van je eigen tekst.

De rest van deze bachelorproef is als volgt opgebouwd:

In Hoofdstuk~\ref{ch:stand-van-zaken} wordt een overzicht gegeven van de stand van zaken binnen het onderzoeksdomein, op basis van een literatuurstudie.

In Hoofdstuk~\ref{ch:methodologie} wordt de methodologie toegelicht en worden de gebruikte onderzoekstechnieken besproken om een antwoord te kunnen formuleren op de onderzoeksvragen.

% TODO: Vul hier aan voor je eigen hoofdstukken, één of twee zinnen per hoofdstuk

In Hoofdstuk~\ref{ch:conclusie}, tenslotte, wordt de conclusie gegeven en een antwoord geformuleerd op de onderzoeksvragen. Daarbij wordt ook een aanzet gegeven voor toekomstig onderzoek binnen dit domein.