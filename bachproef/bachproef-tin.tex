%===============================================================================
% LaTeX sjabloon voor de bachelorproef toegepaste informatica aan HOGENT
% Meer info op https://github.com/HoGentTIN/bachproef-latex-sjabloon
%===============================================================================

\documentclass{bachproef-tin}

\usepackage{hogent-thesis-titlepage}
\usepackage{epigraph}
\setlength\epigraphrule{0pt}
% Packages van de student
\usepackage{subcaption}
\usepackage{pdfpages}
\usepackage[overload]{textcase}
\usepackage[nodayofweek]{datetime}

\usepackage[acronym]{glossaries}
\makeglossaries
\newacronym{acr:ux}{UX}{User Experience}
\newacronym{acr:ui}{UI}{User Interface}
\newacronym{acr:uxd}{UXD}{User Experience Design}
\newacronym{acr:gui}{GUI}{Grafische User Interface}
\newacronym{acr:cli}{CLI}{Command Line Interface}
\newacronym{acr:sus}{SUS}{System Usability Scale}
\newacronym{acr:roi}{ROI}{Return on Investment}
\newacronym{acr:cta}{CTA}{Call-To-Action}
\newacronym{acr:faq}{FAQ}{Frequently Asked Questions}

\usepackage{color}
\definecolor{darkgreen}{rgb}{0,0.6,0}

\usepackage{listings}
\lstset{
    language=R,
    basicstyle=\footnotesize,
    numbers=left,
    numberstyle=\tiny\color{gray},
    stepnumber=1,
    numbersep=5pt,
    frame=single,
    rulecolor=\color{black},
    tabsize=2,
    captionpos=b,
    breaklines=true,
    keywordstyle=\color{blue},
    commentstyle=\color{darkgreen},
    stringstyle=\color{purple},
    escapeinside={\%*}{*)}
} 

%%---------- Documenteigenschappen ---------------------------------------------

\title{Een gebruiker wegwijs maken doorheen een applicatie van aanzienlijke omvang}

\author{Jakob Lierman}

\promotor{Karine Samyn}

\copromotor{Bert Maurau}

\instelling{Cardify}

\academiejaar{2019-2020}

\examenperiode{2}

%===============================================================================
% Inhoud document
%===============================================================================

\begin{document}

%---------- Titelblad ----------------------------------------------------------
\inserttitlepage

%---------- Samenvatting, voorwoord --------------------------------------------
\usechapterimagefalse
%%=============================================================================
%% Voorwoord
%%=============================================================================

\chapter*{\IfLanguageName{dutch}{Woord vooraf}{Preface}}
\label{ch:voorwoord}

Deze bachelorproef werd geschreven in het kader van het behalen van het diploma ``Bachelor in de Toegepaste Informatica'', afstudeerrichting Mobile Apps.

Mijn interesses gaan verder dan enkel de technische kant van mijn opleiding. Net daarom heb ik voor dit onderwerp gekozen. Het verkennen van verschillende \acrshort{acr:ux} en \acrshort{acr:ui}-elementen sprak me enorm aan. Het was een leerrijke ervaring om applicaties en IT voor te leggen aan iedereen in mijn omgeving, jong en oud, om dan van dichtbij te observeren hoe men met de technologie omspringt.

Deze scriptie zou niet tot stand zijn gekomen zonder enkele personen. Hierbij wil ik van dit voorwoord graag gebruik maken om deze personen te bedanken. 

Zonder Bert Maurau, mijn co-promotor, had ik deze scriptie niet kunnen aanvullen met gegevens en voorbeelden vanuit Cardify. Hij en Seppe Vereecken stonden steeds klaar om mijn vragen te beantwoorden en deze scriptie aan te vullen met nuttige info en voorbeelden.

Deze scriptie werd geschreven tijdens de Covid-19-crisis. Bram Van de velde heeft tools aangeboden om mijn proef zo vlot mogelijk te laten verlopen vanop afstand. Zo kon ik de proof-of-concept applicatie delen met iPhone-gebruikers zonder hiervoor met de participant in contact te komen.

Quentin Braet gaf deze scriptie een extra dimensie door zijn kennis van bij In The Pocket te delen.

Karine Samyn, mijn promotor, volgde mijn voortgang nauw op en gaf op regelmatige basis feedback. Door deze opbouwende kritiek werd deze scriptie zonder veel problemen tot een goed einde gebracht.

Amber Priem hielp me doorheen de statistische analyses, waardoor ik mijn gemeten waardes kon omzetten in een conclusie. Ook was ze steeds de eerste die klaar stond om een tekstblok na te lezen of te controleren op grammaticale fouten.

Ten slotte had ik graag nog mijn ouders bedankt voor de financiële steun om deze opleiding tot een goed einde te brengen. Uiteraard had ik hen en iedereen die mijn scriptie even doornam graag nogmaals bedankt om deze tekst na te lezen, ook al ligt deze toch wel uit hun interessegebied.

Ik wens u veel leesplezier toe.

Jakob Lierman

Gent, \shortdate\today



%%=============================================================================
%% Samenvatting
%%=============================================================================

% TODO: De "abstract" of samenvatting is een kernachtige (~ 1 blz. voor een thesis) synthese van het document.

% Deze aspecten moeten zeker aan bod komen:
% - Context: waarom is dit werk belangrijk?
% - Nood: waarom moest dit onderzocht worden?
% - Taak: wat heb je precies gedaan?
% - Object: wat staat in dit document geschreven?
% - Resultaat: wat was het resultaat?
% - Conclusie: wat is/zijn de belangrijkste conclusie(s)?
% - Perspectief: blijven er nog vragen open die in de toekomst nog kunnen onderzocht worden? Wat is een mogelijk vervolg voor jouw onderzoek?

\chapter*{\IfLanguageName{dutch}{Samenvatting}{Abstract}}

Elke gebruiker moet leren werken met de applicaties die deze op zijn of haar toestel installeert. Men kan deze gebruiker helpen door middel van implementatie van learnability-elementen doorheen de software.

De grote meerderheid aan bedrijven en ontwikkelaars hebben echt de tijd of middelen niet om deze technieken zo geoptimaliseerd mogelijk uit te werken. Deze scriptie geeft duidelijkheid over de verschillende mogelijkheden en de effecten ervan. Dit schrijven werd aangevuld met inzichten van ontwikkelaars.

Om het effect van learnability-elementen te meten werd er een usability test uitgevoerd met enkele participanten. Uit dit onderzoek resulteert dat het gebruik van deze elementen wel degelijk voordelig is, als en slechts als deze op de juiste manier verwerkt zijn doorheen de software. Elke software verschilt en voor elke software moet opnieuw bekeken worden waar men best hulp kan voorzien. Dit gaat vlot aan de hand van usability tests doorheen fases van de ontwikkeling.

De relatie tussen (het gebrek aan) in-app user training en de gebruiksduur en/of levensduur van de applicatie wordt aanbevolen als toekomstig onderzoek. Hierbij kan men deze proef als startpunt gebruiken met een gerichte applicatie en een steekproef waarbij elke participant interesse toont in de functionaliteiten van de applicatie.


%---------- Inhoudstafel -------------------------------------------------------
\pagestyle{empty}
\tableofcontents
\cleardoublepage
\pagestyle{fancy}

%---------- Lijst figuren, afkortingen, ... ------------------------------------

\listoffigures
\listoftables
\printacronyms

%---------- Kern ---------------------------------------------------------------

%%=============================================================================
%% Inleiding
%%=============================================================================

\chapter{\IfLanguageName{dutch}{Inleiding}{Introduction}}
\label{ch:inleiding}

\epigraph{If the user can't use it, it doesn't work.}{\textit{Susan Dray}}

In digitale tijden als deze is de vraag naar nieuwe software groot. Programmeurs en IT-bedrijven hebben hun handen vol. Zowat elke sector wil mee zijn met de digitale boot. Je komt hedendaags overal software tegen; de computer op het werk, de smartphone in je broekzak, het bedieningspaneel van een grote kraan, de kassa in de supermarkt, $\dots$ Je gebruikt hoogstwaarschijnlijk tal van software in het dagelijkse leven. Maar hoe leer je nu het best omgaan met deze computerprogramma's werken? Programma's bevatten vaak heel veel functies. Veel van deze functies blijven echter onbenut omdat de gebruiker niet voldoende begrijpt hoe deze functies in zijn werking treden. De programmeurs achter deze software hebben vaak hun handen al vol met het ineen knutselen van al deze functionaliteit dat zij zelf geen tijd hebben om te kijken hoe de eindgebruiker met deze functionaliteit omspringt.

\epigraph{How do I explain what I do at a party? The short version is that I say I humanize technology.}{\textit{Fred Beecher, Director of UX, The Nerdery}}

Hier komt de UX-designer aan bod. Een UX-designer zorgt er voor dat software bruikbaar is voor de eindgebruiker. Taken die zijn job omschrijven omvatten, maar zijn niet beperkt tot maken van prototypes, testen van (deel)producten bij gebruikers, observeren van gedrag van gebruikers op bepaalde functionaliteiten en stukken software, \textit{user flows} creëren en ook onderzoek doen naar de doelgebruiker~\autocite{White2020}. De UX-designer zal dus ook een flow creëren dat ervoor zorgt dat de gebruiker alle functionaliteit van de applicatie goed begrijpt en snel onder de knie heeft.

Eén van de bekendste implementaties hiervan is de ``onboarding''. Je komt vaak in contact met onboarding wanneer je de applicatie voor de eerste maal opstart. Zo'n onboarding kan zeer verschillend zijn van applicatie tot applicatie.
Er bestaan uiteraard meer manieren om de gebruiker de weg te wijzen doorheen software. Een simpele \textit{tooltip} of zelfs een help-pagina of leerplatform doet ook wonderen.

\section{\IfLanguageName{dutch}{Probleemstelling}{Problem Statement}}
\label{sec:probleemstelling}

Bij veel software-ontwikkelaars en -designers stelt de vraag zich frequent als een bepaalde implementatie van onboarding of in-app user training wel het gewenste effect bekomt. Er bestaan verscheidene manieren om dit in een applicatie te verwerken, om te weten als de ene manier een beter resultaat boekt dan de andere is een onderzoek nodig.

We bekeken de voordelen van onboarding en in-app user training nu voornamelijk vanuit het oogpunt van de eindgebruiker. Echter kan dit ook effect hebben op andere aspecten binnen een bedrijf. Deze scriptie zal ook bekijken indien de snelheid en het gemak van het leren van de functionaliteiten in software een grote invloed op het klantbehoud. Er zal dus onderzocht worden indien de klant zich meer geneigd gaat voelen een bepaalde applicatie met een andere en/of betere implementatie van de verschillende technieken frequenter te gebruiken.

\section{\IfLanguageName{dutch}{Onderzoeksvraag}{Research question}}
\label{sec:onderzoeksvraag}

\subsection{Hoofdonderzoeksvraag}
\label{sec:hoofdonderzoeksvraag}

Zoals reeds aangehaald in sectie~\ref{sec:probleemstelling} zal dit onderzoek zich focussen op implementaties van onboarding en in-app user training en de gevolgen van deze implementaties. Daaruit vloeit volgende hoofdonderzoeksvraag voort:

\begin{itemize}
    \item Kan een (betere) onboarding en in-app user training ervoor zorgen dat de eindgebruiker een beter inzicht heeft op de totale functionaliteit van een grote applicatie?
\end{itemize}

\subsection{Deelonderzoeksvragen}
\label{sec:deelonderzoeksvragen}

Ter ondersteuning van de hoofdonderzoeksvraag zijn er ook nog enkele deelonderzoeksvragen opgesteld:

\begin{itemize}
    \item Hoe een grote hoeveelheid aan functionaliteiten beheersbaar houden voor de eindgebruiker?
    \item Heeft (het gebrek aan) in-app user training effect op de gebruiksduur en/of levensduur van de applicatie?
    \item Hoe de eindgebruiker wegwijs maken in een grote applicatie?
\end{itemize}

Doorheen deze scriptie zal op deze deelonderzoeksvragen een antwoord geformuleerd worden.

\section{\IfLanguageName{dutch}{Onderzoeksdoelstelling}{Research objective}}
\label{sec:onderzoeksdoelstelling}

Het hoofddoel van dit onderzoek is het aantonen van de werking en gevolgen van verschillende technieken om de eindgebruiker familiair te maken met de applicatie. Door middel van een proof-of-concept zal worden aangetoond welke technieken het meest gewenste resultaat hebben op de eindgebruiker.

Een tweede doel bestaat uit het onderzoeken als (het gebrek aan) deze technieken ook effect heeft op het klantbehoud bij de applicatie.

Een laatste doel is om er voor te zorgen dat ontwikkelaars, die dit lezen, een idee krijgen van de UX-technieken die men kan gebruiken bij de implementatie van hun applicatie. Alsook UX-designers kunnen deze scriptie raadplegen bij het uitdenken van hun user flows.

\section{\IfLanguageName{dutch}{Opzet van deze bachelorproef}{Structure of this bachelor thesis}}
\label{sec:opzet-bachelorproef}

% Het is gebruikelijk aan het einde van de inleiding een overzicht te
% geven van de opbouw van de rest van de tekst. Deze sectie bevat al een aanzet
% die je kan aanvullen/aanpassen in functie van je eigen tekst.

De rest van deze bachelorproef is als volgt opgebouwd:

In Hoofdstuk~\ref{ch:stand-van-zaken} wordt een overzicht gegeven van de stand van zaken binnen het onderzoeksdomein, op basis van een literatuurstudie.

In Hoofdstuk~\ref{ch:methodologie} wordt de methodologie toegelicht en worden de gebruikte onderzoekstechnieken besproken om een antwoord te kunnen formuleren op de onderzoeksvragen.

% TODO: Vul hier aan voor je eigen hoofstukken, één of twee zinnen per hoofdstuk

In Hoofdstuk~\ref{ch:conclusie}, tenslotte, wordt de conclusie gegeven en een antwoord geformuleerd op de onderzoeksvragen. Daarbij wordt ook een aanzet gegeven voor toekomstig onderzoek binnen dit domein.
\chapter{\IfLanguageName{dutch}{Stand van zaken}{State of the art}}
\label{ch:stand-van-zaken}

% Tip: Begin elk hoofdstuk met een paragraaf inleiding die beschrijft hoe dit hoofdstuk past binnen het geheel van de bachelorproef. Geef in het bijzonder aan wat de link is met het vorige en volgende hoofdstuk.

Zoals uit vorig hoofdstuk kan worden afgeleid zal deze scriptie onderzoek doen naar de implementatie en effecten van bepaalde UX en UI-elementen. Maar alvorens van start te gaan hiermee zal eerst gekeken worden naar wat de algemene rol is van UX en UI in software-ontwikkeling.

\section{User experience in software}
\label{sec:user-experience-in-software}

In het traditioneel proces van het ontwikkelen van software staat de functionaliteit centraal. Ontwikkelaars bekijken alle vereisten en starten met de belangrijkste. Functionaliteit krijgt hier doorgaans de voorkeur. \textcite{Harutyunyan2019} stelden vast dat dit de laatste jaren echter aan het wijzigen is. De traditionele softwareontwikkeling is plaats aan het maken voor softwareontwikkeling met user experience in het achterhoofd. Dit fenomeen noemt men User Experience Design of kortweg UXD. Omdat de term UXD in de literatuur nog sterk evolueert heeft dit nog geen algemeen aanvaarde definitie. Men kan stellen dat user experience design een proces is waarbij men gebruiksgedrag zal manipuleren aan de hand van de bruikbaarheid en wenselijkheid in de interactie met een product. 

Men doet al lang onderzoek naar user experience in software. Zo toonden \textcite{Carroll1984} het belang van training in complexe systemen al aan anno 1984. Deze training van gebruikers behandelen we later in dit hoofdstuk.

\begin{figure}[h]
    \centering
    \def\svgwidth{.8\columnwidth}
    \input{./img/user-experience-waarom-wat-hoe.pdf_tex}
    \caption{Het waarom, wat en hoe van user experience design}
    \label{fig:ux-waarom-wat-hoe}
\end{figure}

Een UX-designer bekijkt het product niet enkel als een het product zelf. Deze persoon analyseert hoe de eindgebruiker het product in gebruik neemt en past het product aan zodat de gebruikservaring optimaal is. De designer neemt het \textit{waarom}, \textit{wat} en \textit{hoe} van productgebruik in acht (zie figuur~\ref{fig:ux-waarom-wat-hoe}) \autocite{Hassenzahl2013}. De \textit{wat} in productgebruik verwijst gewoonlijk naar wat een gebruiker kan doen door middel van het product. Dit is bijvoorbeeld ``een foto maken'' of ``een spel kopen''. De \textit{hoe} staat dan ook effectief voor de gebruiker het product gebruikt. Dit is meer op een operationeel niveau zoals het navigeren door software met behulp van knoppen en andere attributen. De designer zal zich voornamelijk focussen op het ``hoe'' van het productgebruik. Dit omvat de gegeven functionaliteit op een aantrekkelijke manier zeer toegankelijk maken.

\section{Belangrijke factoren bij user experience}
\label{sec:belangrijke-factoren-bij-user-experience}

Er zijn talloze factoren die ervoor zorgen dat men op een verschillende manier naar dezelfde applicatie moet kijken. Zo zijn de gebruikers allemaal verschillend, maar er moet ook rekening gehouden worden met verschillende omgevingen, culturen, enz. Een toestel om parameters op te meten bij schepen moet dus waterbestendig zijn. Een kindvriendelijke tablet is best schokbestendig. Een applicatie om notities te maken bij vergaderingen is best geluidsloos.

De factoren kunnen gegroepeerd worden in vijf groepen (zie figuur~\ref{fig:ux-factoren}). Culturele factoren omvatten bijvoorbeeld religie, taal en gewoontes. Zo moet bij het ontwerpen van een website gericht op asielzoekers bijvoorbeeld rekening houden met het gebrek aan kennis van de landstaal.

Een mobiele applicatie waarbij de gebruiker een vervoersbewijs moet voorleggen op een voertuig van het openbaar vervoer zal bijvoorbeeld rekening moeten houden met het feit dat de sociale factor tijdsdruk hier belangrijk is. Indien deze applicatie niet tijdig het vervoersbewijs laat zien zal er een hele wachtrij ontstaan die dan vertragingen tot gevolg heeft.

Je kan factoren met betrekking tot de context waarin het product gebruikt wordt opvatten als bijvoorbeeld tijd en locatie. Bij het bestellen van een pakket krijg je vaak een tracking-link waarbij ook het tijdstip van levering staat. Een internationale leverancier moet dus zeker voorzien dat de tijd van de levering in de juiste tijdzone weergegeven wordt.

De gebruiker zelf verschilt uiteraard ook. Een applicatie gericht op een ouder publiek voorziet best grote tekst en duidelijke iconen.

Het product zelf moet uiteindelijk ook nog bruikbaar zijn en alle functionaliteiten moeten eenvoudig bereikbaar zijn.
Een hele boterham voor de user experience designer om onderzoek naar te doen voor zijn use case.

\begin{figure}
    \centering
    \def\svgwidth{.8\columnwidth}
    \input{./img/user-experience-factoren.pdf_tex}
    \caption{Belangrijke factoren bij user experience}
    \label{fig:ux-factoren}
\end{figure}

% TODO

\chapter{\IfLanguageName{dutch}{Methodologie}{Methodology}}
\label{ch:methodologie}

% TODO: Hoe ben je te werk gegaan? Verdeel je onderzoek in grote fasen, en licht in elke fase toe welke stappen je gevolgd hebt. Verantwoord waarom je op deze manier te werk gegaan bent. Je moet kunnen aantonen dat je de best mogelijke manier toegepast hebt om een antwoord te vinden op de onderzoeksvraag.

% TODO - X vervangen

In dit hoofdstuk zal besproken worden hoe het experiment in zijn werk gaat, hoe de steekproef werd getrokken alsook welke variabelen er gemeten werden.

\section{Het experiment}
\label{sec:experiment}

Om te bekijken indien een implementatie van bepaalde learnability technieken invloed heeft op de eindgebruiker zullen er usability tests worden uitgevoerd bij testpersonen. Welke deze testpersonen zijn en hoe er participanten werden verzameld is te raadplegen in hoofdstuk~\ref{sec:experiment:populatie-steekproef}.

Bij dit experiment zal er vooral gefocust worden op laboratory testing (zie hoofdstuk~\ref{sec:usability-testing:lab-field-testing}). Dit wil zeggen dat de testpersoon de applicatie zal testen in het bijzijn van een moderator, maar niet strikt noodzakelijk in de omgeving waarin de applicatie in een reëel scenario gebruikt zou worden. De moderator noteert hierbij alle bevindingen van de testpersoon, alsook waar deze eventueel moeilijkheden ondervindt.

Het experiment zal uitgevoerd worden bij twee groepen. De ene groep krijgt een proof-of-concept applicatie waarin alle learnability technieken in verwerkt zijn. De controlegroep krijgt een applicatie zonder deze learnability elementen. Tijdens deze usability tests wordt van de testpersonen verwacht dat deze een reeks taken tot een goed einde proberen te brengen. Gedurende het experiment worden een aantal variabelen gemeten (zie hoofdstuk~\ref{sec:experiment:variabelen}). De procedure die gevolgd wordt in dit onderzoek is te raadplegen in hoofdstuk~\ref{sec:test-afnemen}.

\subsection{De populatie en steekproef}
\label{sec:experiment:populatie-steekproef}

Aan dit experiment kan quasi iedereen deelnemen indien deze een basiskennis hebben van een smartphone. Leeftijd, afkomst, achtergrond, kennis en andere factoren zijn van weinig belang. Er moet echter wel voldoende variatie zijn. Wanneer de groep participanten enkel uit jonge personen met een technische achtergrond bestaat zal het resultaat sterk beïnvloed worden. Het is dus de bedoeling om zowel jong, oud, technisch onderlegd en technisch leek op te nemen in de poule van participanten.

Participanten worden gezocht zowel in de kring van familie en vrienden als daarbuiten. De resultaten van de laboratory tests met deze participanten gingen normaal aangevuld worden met resultaten uit guerilla tests (zie hoofdstuk~\ref{sec:usability-testing:testmethoden:guerilla}). Deze scriptie werd opgesteld tijdens de Covid-19-situatie en het was dus niet mogelijk en/of veilig om guerilla testing uit te voeren.

Omdat uit veiligheidsoverwegingen geopteerd werd om de test op afstand uit te voeren (de moderator is dus niet fysiek aanwezig bij de testpersoon) moet de testpersoon de proof-of-concept applicatie installeren op hun eigen toestel. De applicatie is geschreven voor het mobiele besturingssysteem van Apple (iOS 13 of hoger) en werd geoptimaliseerd voor gebruik op een smartphone (iPhone of iPod touch). Bij het verzamelen van gegevens van potentiële participanten werd er gevraagd indien deze in het bezit zijn van een toestel dat aan deze voorwaarden voldoet.

Een deelnameformulier werd via verschillende kanalen (sociale netwerken, mond-tot-mondreclame, ...) verspreid om zoveel mogelijk potentiële participanten te bereiken. Het deelnameformulier werd opgesteld in \href{https://forms.office.com/}{Microsoft Forms}. Het volledige formulier is bijgevoegd in bijlage~\ref{bijlage:deelnameformulier}.

In totaal werd het formulier x keer ingevuld. Daaruit werden X personen geselecteerd om deel te nemen aan de test. In figuur~X is te zien in welke leeftijdsgroep de participanten zich bevinden, welk geslacht deze hebben en indien ze zichzelf al dan niet zien als technisch onderlegd.


% TODO - X vervangen
In totaal werd het formulier x keer ingevuld. Daaruit werden x personen geselecteerd om deel te nemen aan de test. In figuur~X is te zien in welke leeftijdsgroep de deelnemers zich bevinden, welk geslacht deze hebben en indien ze zichzelf al dan niet zien als technisch onderlegd.


\chapter{\IfLanguageName{dutch}{Over de metingen}{About the measurements}}
\label{ch:metingen}

\section{De gebruikstijd bij bepaalde opdrachten}
\label{sec:gebruikstijd}

\section{Het vragen om hulp}
\label{sec:vragen-hulp}

\section{Gebruik van de voorziene functionaliteiten}
\label{sec:gebruik-functionaliteiten}

\section{De SUS-score}
\label{sec:sus}

\section{Voorkeur voor onboarding en help-elementen}
\label{sec:voorkeur-onboarding}

\section{Voorkeur voor help-sectie}
\label{sec:voorkeur-help}

\section{De gebruiksduur en/of levensduur van de applicatie}
\label{sec:gebruiksduur}


\chapter{\IfLanguageName{dutch}{Resultaten van de proef}{Results of the test}}
\label{ch:resultaten}

\section[Onderzoeksvraag 1]{Onderzoeksvraag 1: Kan een (betere) onboarding en in-app user training ervoor zorgen dat de eindgebruiker een beter inzicht heeft op de totale functionaliteit van een grote applicatie?}
\label{sec:onderzoeksvraag-1}

Om een antwoord te formuleren op de eerste onderzoeksvraag kan men een aantal afhankelijke variabelen in overweging nemen. Enerzijds kan men kijken naar alle tijden op de zes taken, anderzijds naar de SUS-score en de vragen met betrekking tot learnability uit de SUS-vragenlijst, maar men kan ook in overweging nemen of de participant al dan niet om hulp vroeg tijdens één van de taken.

Deze laatste variabelen, namelijk de zes dichotome hulp-gevraagd variabelen, werden onderzocht met behulp van een $\chi^2$-test (chi-kwadraattest). In deze test werd de groep waarvan de participant deel uitmaakt als onafhankelijke variabele opgenomen. De resultaten zijn waar te nemen in tabel~\ref{tab:chisq-hulp}. Voor de opdrachten \textit{instellingen}, \textit{spaardoel toevoegen}, \textit{bedrag toevoegen} en \textit{berekening} hebben alle participanten het doel bereikt zonder hulp.

\begin{table}[]
    \centering
    \begin{tabular}{r|cccc}
        & $\chi^2$ & $df$ & $p$ & $N$ \\ \hline
        Spaardoel verwijderen & $5.16$ & $1$ & $0.02$ & $25$ \\
        Groot bedrag toevoegen & $0.003$ & $1$ & $0.95$ & $25$
    \end{tabular}
    \caption{$\chi^2$ resultaten indien de participant hulp nodig had}
    \label{tab:chisq-hulp}
\end{table}

Uit de resultaten van de $\chi^2$-test valt af te leiden dat er een significant verband is tussen de twee variabelen bij de opdracht \textit{spaardoel verwijderen}, $\chi^2 (1, N = 25) = 5.16$, $p = 0.02$, maar niet bij de opdracht \textit{groot bedrag toevoegen}, $\chi^2 (1, N = 25) = 0.003$, $p = 0.95$, $ns$. Bij enkele functionaliteiten zal de gebruiker dus minder hulp nodig hebben wanneer men bepaalde vormen van onboarding implementeert in de applicatie. Echter zal de gebruiker niet bij elke functionaliteit een voordeel halen uit de onboarding. Om te weten wanneer de onboarding van pas komt kan men best de applicatie laten testen door enkele personen en noteren waar deze problemen ondervinden.

Om de overige afhankelijke variabelen op te nemen in de analyses, werd eerst een $t$-test uitgevoerd. In deze $t$-test werd opnieuw als onafhankelijke variabele de groep waarvan de participant deel uitmaakt opgenomen. Als afhankelijke variabele werd gekozen voor een gemiddelde van de tijden op alle taken. Participanten voltooiden de opdrachten sneller wanneer deze de proof-of-concept applicatie hadden waarbij de onboarding en help-elementen beschikbaar waren ($M = 23.36$, $SD = 9.95$) in vergelijking met wanneer deze geen onboarding en help-elementen ter beschikking hadden ($M = 43.57$, $SD = 21.76$), $t(24.033) = -8.426$, $p < .001$.

Deze analyse toont een verschil aan tussen wanneer men wel of geen gebruik kon maken van de learnability-elementen, maar de assumptie wordt gemaakt dat het effect van learnability-elementen op de zes taken gelijkaardig genoeg is dat er een gemiddelde van kan worden genomen. Omwille van deze reden wordt voor elke taak apart nog eens een $t$-test uitgevoerd. De resultaten van deze $t$-test zijn waar te nemen in tabel~\ref{tab:ttest-opdrachten}. Per opdracht is ook hier duidelijk dat participanten die gebruik maakten van onboarding en help-elementen significant minder tijd nodig hadden om de opdracht te voltooien. De gemiddelde tijden zijn te vinden in tabellen~\ref{tab:beschrijving-tijden-zonder-elementen} en \ref{tab:beschrijving-tijden-met-elementen}.

\textit{De berekening van deze waarden in R is bijgevoegd in bijlage~\ref{bijlage:r-1}.}

\begin{table}[]
	\centering
	\begin{tabular}{r|ccc}
		\textbf{Opdracht} & \textbf{$t$} & \textbf{$df$} & \textbf{$p$} \\ \hline
		Instellingen & $-8.55$ & $24.10$ & $< 0.001$ \\
		Spaardoel toevoegen & $-11.11$ & $24.04$ & $< 0.001$ \\
		Bedrag toevoegen & $-9.27$ & $24.05$ & $< 0.001$ \\
		Spaardoel verwijderen & $-4.19$ & $24.01$ & $< 0.001$ \\
		Berekening & $-10.55$ & $24.05$ & $< 0.001$ \\
		Groot bedrag toevoegen & $-5.27$ & $24.01$ & $< 0.001$ \\
		\textit{SUS-score} & $-29.36$ & $24.07$ & $< 0.001$
	\end{tabular}
	\caption{$t$-testen van alle opdrachten}
	\label{tab:ttest-opdrachten}
\end{table}

Een betere onboarding kan er zeker en vast voor zorgen dat de eindgebruiker sneller met de applicatie overweg kan. Hoe goed deze applicatie is in de ogen van de gebruiker hangt echter van meerdere variabelen af. Zo is een matige applicatie met een goede learnability niet rechtstreeks een betere applicatie. Waar en wanneer er in-app help elementen moeten geïmplementeerd worden hangt sterk af van de gebruiker. Wat vaak werd opgemerkt bij het afnemen van deze proef is dat elke gebruiker verschillend is en de ene gebruiker een bepaalde functionaliteit begrijpt zonder hulp terwijl de andere gebruiker sterk leunt op de hulp. Bij het bouwen van een applicatie moet dus zeker rekening gehouden worden met het doelpubliek bij het implementeren van onboarding en help-elementen. Usability tests voor en na die implementatie zijn sterk aan te raden voor betere inzichten.

\section[Onderzoeksvraag 2]{Onderzoeksvraag 2: Hoe een grote hoeveelheid aan functionaliteiten beheersbaar houden voor de eindgebruiker?}
\label{sec:onderzoeksvraag-2}

Het beheersbaar houden van een aanzienlijke hoeveelheid aan functionaliteiten is niet enkel een taak voor de onboarding en andere learnability elementen. Dit is een breder probleem waarvoor men idealiter een verder onderzoek in het \acrshort{acr:ux}-vakgebied houdt. Zoals in hoofdstuk~\ref{sec:learnability}, \ref{sec:onboarding} en \ref{sec:in-app-training} te lezen valt, is het voorzien van bepaalde onboarding en help-elementen wel een stap in de goede richting. Bij het afnemen van de proef viel ook op te merken dat voor de herhaling van de opdracht waar men een bedrag moest toevoegen de participant zelfverzekerder was wanneer deze de proof-of-concept applicatie had waarin de learnability-elementen aanwezig waren. Zo waren minder twijfels merkbaar.

Waarop zeker gefocust moet worden bij de implementatie van learnability-elementen in software van een aanzienlijke omvang is dat men de gebruiker initieel niet overspoelt met uitleg. De gebruiker verliest zo snel de aandacht en zal niet alles kunnen onthouden. Hotspots (zie hoofdstuk~\ref{sec:onboarding:hotspots}) zijn zo vaak een betere keuze dan een rondleiding (zie hoofdstuk~\ref{sec:onboarding:rondleidingen}), omdat de gebruiker zo zelf kiest wanneer deze wil starten met de uitleg.

Men kan de beheersbaarheid meten met de vragen uit de SUS-vragenlijst die betrekking hebben tot learnability. Participanten antwoorden gunstiger op deze vragen wanneer deze de proof-of-concept applicatie hadden waarbij de onboarding en help-elementen beschikbaar waren (vraag 4: $M = 1.31$, $SD = 0.85$, vraag 10: $M = 1.62$, $SD = 1.04$) in vergelijking met wanneer deze geen onboarding en help-elementen ter beschikking hadden (vraag 4: $M = 1.67$, $SD = 0.78$, vraag 10: $M = 1.83$, $SD = 0.83$), vraag 4: $t(40.07) = -4.96$, $p < .001$, vraag 10: $t(37.09) = -5.63$, $p < .001$.

\textit{De berekening van deze waarden in R is bijgevoegd in bijlage~\ref{bijlage:r-2}.}

\section[Onderzoeksvraag 3]{Onderzoeksvraag 3: Heeft (het gebrek aan) in-app user training effect op de gebruiksduur en/of levensduur van de applicatie?}
\label{sec:onderzoeksvraag-3}

Indien de gebruiker de applicatie voor een langdurige periode zou gebruiken is moeilijk te voorspellen. De belangrijkste fase voor de gebruiksduur is het initieel gebruik. Wanneer de gebruiker niet overtuigd is van de applicatie zal hij of zij die sneller terug verwijderen. Als afsluitende vraag in de proef werd aan de participanten gevraagd indien ze deze applicatie zouden houden op hun toestel of indien ze deze zouden verwijderen of vergelijken met concurrerende software.

In figuur~\ref{fig:beschrijving-would-keep} is te merken dat er eventueel een verschil zal zijn tussen de groep met en zonder onboarding en usability elementen. Na uitvoering van de $\chi^2$-test is er echter geen significante relatie merkbaar tussen het al dan niet aanwezig zijn van learnability-elementen en indien de participant de applicatie zou houden op zijn of haar persoonlijk toestel, $\chi^2 (1, N = 25) = 1.92$, $p = 0.17$, $ns$. De gebruiker die toegang had tot onboarding en help-elementen was dus niet meer geneigd de applicatie te houden dan de gebruiker die hiertoe geen toegang had. Een verklaring hiervoor kan zijn dat een hele hoop meer factoren hierop inspelen. Deze proef is bijvoorbeeld uitgevoerd bij participanten waarvan niet iedereen op zoek was naar een applicatie met deze functionaliteit. Verder onderzoek kan hier meer duidelijkheid en een gunstiger resultaat bieden.

\textit{De berekening van deze waarden in R is bijgevoegd in bijlage~\ref{bijlage:r-3}.}

\section[Onderzoeksvraag 4]{Onderzoeksvraag 4: Hoe de eindgebruiker wegwijs maken in een grote applicatie?}
\label{sec:onderzoeksvraag-4}

Hoe een gebruiker omgaat met de functionaliteiten binnen de applicatie is één ding, maar de gebruiker moet uiteraard ook overweg kunnen met de navigatie binnen deze applicatie om zo bij de gewenste functionaliteiten uit te komen. Net zoals bij onderzoeksvraag 2 (zie hoofdstuk~\ref{sec:onderzoeksvraag-2}) is dit een breder probleem uit het \acrshort{acr:ux}-vakgebied. Participanten uit deze proef hebben wel aangetoond dat deze het handig en bruikbaar vonden dat bij de initiële opstart van de applicatie er een uitleg was voorzien voor de elementen in de tab-bar (zie figuur~\ref{fig:piggy:tabbar}). Een uitleg voorzien zorgt ervoor dat de iconen uit de navigatie een duidelijke betekenis krijgen. Hierdoor gaat de navigatie vlotter.

\begin{figure}[h!]
    \centering
    \includegraphics[width=.5\columnwidth]{piggy-tabbar}
    \caption{Navigatie in de proof-of-concept applicatie}
    \label{fig:piggy:tabbar}
\end{figure}

Om de gebruiker niet aan zijn of haar lot over te laten is het mogelijk om deze in de onboarding al enkele taken te laten voltooien vooraleer aan het echte werk te beginnen. In de proof-of-concept applicatie was het mogelijk om een naam in te stellen, dit kon zowel in de instellingen als in de onboarding (zie figuur~\ref{fig:piggy:name}). In de proef viel op dat enkel participanten die de proof-of-concept applicatie met onboarding hadden de naam ook effectief wijzigden (zie figuur~\ref{fig:beschrijving-changed-name}). De relatie tussen het instellen van een naam en de aanwezigheid onboarding-elementen is significant, $\chi^2 (1, N = 25) = 25$, $p = < .001$. De gebruiker is dus meer geneigd bepaalde opdrachten te voltooien indien deze de kans krijgt om dit in de onboarding van de applicatie te doen.

\textit{De berekening van deze waarden in R is bijgevoegd in bijlage~\ref{bijlage:r-4}.}

\begin{figure}[h!]
	\centering
	\subfloat[Instellingen]{
		\includegraphics[width=.27\linewidth]{piggy-name-settings-1}
		\label{fig:piggy:name:settings}
	}
	\qquad
	\subfloat[Naam wijzigen in instellingen]{
		\includegraphics[width=.27\linewidth]{piggy-name-settings-2}
		\label{fig:piggy:name:settings-2}
	}
	\qquad
	\subfloat[Naam wijzigigen in onboarding]{
		\includegraphics[width=.27\linewidth]{piggy-name-onboarding}
		\label{fig:piggy:name:onboarding}
	}
	\caption{De naam van de gebruiker wijzigen in de proof-of-concept applicatie}
	\label{fig:piggy:name}
\end{figure}

\begin{figure}[h]
	\centering
	\includegraphics[width=.46\columnwidth]{beschrijving-changed-name}
	\caption{Indien de participant de naam wijzigde bij het verkennen van de applicatie}
	\label{fig:beschrijving-changed-name}
\end{figure}


%%=============================================================================
%% Conclusie
%%=============================================================================

\chapter{Conclusie}
\label{ch:conclusie}

% TODO: Trek een duidelijke conclusie, in de vorm van een antwoord op de onderzoeksvra(a)g(en). Wat was jouw bijdrage aan het onderzoeksdomein en hoe biedt dit meerwaarde aan het vakgebied/doelgroep? Reflecteer kritisch over het resultaat. In Engelse teksten wordt deze sectie ``Discussion'' genoemd. Had je deze uitkomst verwacht? Zijn er zaken die nog niet duidelijk zijn? Heeft het onderzoek geleid tot nieuwe vragen die uitnodigen tot verder onderzoek?

In dit onderzoek werd een antwoord gegeven op de onderzoeksvraag ``Kan een (betere) onboarding en in-app user training ervoor zorgen dat de eindgebruiker een beter inzicht heeft op de totale functionaliteit van een grote applicatie?''. Hiervoor werd een proof-of-concept applicatie opgesteld met en zonder learnability-elementen. Deze werd getest door twee groepen participanten.

De resultaten van deze proef gaven aan dat de learnability-elementen zeker hun nut hebben. Echter moet men zeer voorzichtig omspringen met waar en wanneer men deze plaatst. Zoals vooraf werd besproken is het belangrijk om zich te beperken tot de essentie, een gebruiker leert de details van de applicatie pas achteraf kennen. Zo wordt de gebruiker niet meteen overspoelt met informatie. Frequent de applicatie laten testen om zo te bekijken waar er hulp moet voorzien worden, is één van de belangrijkste stappen bij het implementeren van learnability in software.

Er bestaan verschillende technieken om onboarding op een correcte manier te implementeren. Niet elke techniek kan voor elke applicatie of use case gebruikt worden. De ontwikkelaar van de software bekijkt best welke technieken geschikt zijn voor zijn of haar specifieke use case. Deze verschillen in \acrshort{acr:ux} voor verschillende applicaties kunnen eventueel onderzocht worden in een verder onderzoek.

De verspreiding van help-elementen doorheen de applicatie is een volgend belangrijk aspect bij de implementatie van onboarding en help-elementen. Door de gebruiker niet initieel te overspoelen met info, maar deze pas geleidelijk aan nieuwe functionaliteiten aan te leren zal de gebruiker de werking van deze functionaliteiten beter onthouden. De gebruiker zal ook minder snel de aandacht verliezen dan tijdens bijvoorbeeld één grote rondleiding.

Er werd voor deze applicatie geen significante relatie gevonden tussen (het gebrek aan) in-app user training en de gebruiksduur en/of levensduur van de applicatie. Wanneer een bedrijf hun software test bij gebruikers die ook effectief een meerwaarde hebben aan de functionaliteiten van deze software kan dit resultaat met hoogste waarschijnlijkheid variëren. Dit kan eventueel bewezen worden in een verder onderzoek.

Een interessante ondervinding uit deze scriptie is het feit dat het mogelijk is de gebruiker aan te zetten tot het gebruik van bepaalde functionaliteiten. Door deze functionaliteiten aan te halen in de onboarding is de gebruiker meer geneigd deze daar uit te proberen en verder te gebruiken. Dit kan interessant zijn om bijvoorbeeld accountcreatie in de kijker te zetten.


%%=============================================================================
%% Bijlagen
%%=============================================================================

\appendix
\renewcommand{\chaptername}{Appendix}

%%---------- Onderzoeksvoorstel -----------------------------------------------

\chapter{Onderzoeksvoorstel}

Het onderwerp van deze bachelorproef is gebaseerd op een onderzoeksvoorstel dat vooraf werd beoordeeld door de promotor. Dat voorstel is opgenomen in deze bijlage.

%---------- Inleiding ---------------------------------------------------------

\section{Introductie} % The \section*{} command stops section numbering
\label{sec:introductie}

Technologie neemt overal de overhand. Meer leeftijdsgroepen maken gebruiken van web- en mobiele applicaties. Deze softwarepakketten worden voor alle mogelijke doeleinden gebruikt, denk maar aan een management platform in een bedrijf, of een mobiel spel om tegen je vrienden te quizzen.

Deze applicaties moeten op een bepaalde manier aan de eindgebruiker uitgelegd worden. Er bestaan verscheidene technieken om alle functionaliteiten binnen de software aan de man te brengen. Boeken deze technieken wel het gewenste resultaat? Is de ene techniek beter dan de andere?

In deze bachelorproef onderzoeken we de onboarding en in-app user training aan de hand van volgende onderzoeksvragen en deelvragen:

\begin{itemize}
    \item Kan een (betere) onboarding en in-app user training ervoor zorgen dat de eindgebruiker een beter inzicht heeft op de totale functionaliteit van een grote applicatie?
    \item Hoe een grote hoeveelheid aan functionaliteiten beheersbaar houden voor de eindgebruiker?
    \item Heeft (het gebrek aan) in-app user training effect op de gebruiksduur en/of levensduur van de applicatie?
    \item Hoe de eindgebruiker wegwijs maken in een grote applicatie?
\end{itemize}

%---------- Stand van zaken ---------------------------------------------------

\section{Literatuurstudie}
\label{sec:state-of-the-art}

\subsection{Wat is 'onboarding'?}
Er bestaan al verscheidene technieken die mogelijks geïmplementeerd kunnen worden in een applicatie om het de gebruiker eenvoudiger te maken deze applicatie te gebruiken. Eén van de bekendste technieken om een gebruiker wegwijs te maken binnen een softwarepakket is 'onboarding'. Onboarding vindt plaats wanneer u de software voor de eerste maal opstart. Het geeft de eindgebruiker een duidelijk inzicht van hoe de applicatie werkt, hoe de functionaliteiten in zijn werk gaan en welke voordelen er worden aangeboden~\autocite{Cardoso2017}.

\subsection{Voorgaand onderzoek}
Bestaand onderzoek bestudeert een manier van onboarding en in-app user training door middel van een 'Learning Center'~\autocite{CamachoHerrero2019}. Hierbij maakt men gebruik van een in-app onderwijsfunctie die de eindgebruiker alle mogelijke functionaliteiten van de software kan uitleggen. Dit platform wil de gebruiker meer betrekken en aanmoedigen om alle elementen te voltooien door middel van gamification elementen zoals checklists of voortgangsbalken. Per onderdeel zelf werd grotendeels gebruik gemaakt van video's die alle uitleg voorzien. In deze bachelorproef zal verder onderzocht worden als het voordeliger is om dit platform op te splitsen in kleine stukken per functionaliteit. Pas wanneer de gebruiker in contact komt met bepaalde functionaliteiten zal de hij de in-app training te zien krijgen.

Onboarding is vaak ook meer dan enkel uitleg over de applicatie. Ook elementen zoals de registratie zijn een stap in het proces om met de software overweg te kunnen~\autocite{renz2014}. Ook een welkomstmail is een veelgebruikt middel om de gebruiker vertrouwd te maken met de applicatie.

Onboarding is verschillend voor webapplicaties als mobiele applicaties~\autocite{RamirezAlvarez2018}. In beide gevallen moet grondig onderzocht worden wat de gebruiker wil bereiken en het gedrag van deze gebruikersgroep is. Applicaties gericht op een jeugdig publiek zullen de onboarding anders aanpakken dan softwarepakketten gericht op het vervullen van een taak in een bedrijfsomgeving.

\subsection{UX testing}
Voorgaand onderzoek maakt duidelijk dat er veel voorbereiding nodig is op vlak van user experience (UX) en usability testing. Om te weten wat de eindgebruiker wil bereiken en welke acties die daarvoor zal ondernemen is er genoeg inlevingsvermogen nodig~\autocite{gualtieri2009}.
Ook mag men er nooit vanuit gaan dat de ontwikkelaar het bij het rechte eind heeft. Hij is immers niet de eindgebruiker en bekijkt de probleemstelling op een andere manier.

%---------- Methodologie ------------------------------------------------------
\section{Methodologie}
\label{sec:methodologie}

Het onderzoeken van technieken zoals in-app user training en bepaalde manieren van het implementeren van deze technieken zal vooral gebeuren aan de hand van experimenten in het kader van usability testing.

Hierbij zal gebruik gemaakt worden van enkele varianten van dummy-applicaties (hetzij met of zonder gebruik van bovenstaande technieken). Gebruikers van de dummy-applicatie zullen fysiek de software doorlopen en enkele opdrachten voltooien. Aan de hand van metingen op basis van een aantal criteria zal kunnen worden besloten of er over een succesvolle test mag gesproken worden. Deze criteria zijn onder meer, maar niet beperkt tot; het succespercentage, het aantal fouten per tijdseenheid, de gemiddelde tijd en de subjectieve tevredenheid van de gebruiker.

De bovengenoemde dummy-applicaties zullen mobiele en/of webapplicaties zijn waarin al dan niet de te onderzoeken technieken geïmplementeerd zullen zijn. Om het onderzoek zo breed mogelijk te houden zal er zowel een applicatie zijn die een software uit de industriële wereld nabootst (denk aan het besturen van een machine, het beheren van bestellingen, ...) als een applicatie voor alledaags gebruik (denk aan een boodschappenlijstje, sociaal netwerk, ...).

Het te onderzoeken doelpubliek zal ook zo breed mogelijk gehouden worden aangezien iedereen ooit wel eens in aanraking komt met software. Het doelpubliek zal nadien opgesplitst worden in doelgroepen op basis van leeftijd en technische kennis.

De gebruikers krijgen enkele opdrachten die ze tot een goed einde moeten brengen op de te testen applicatie. Aan de hand van usability testing zal gemeten worden hoe kwaliteitsvol de software was. De belangrijkste criteria bij deze meting zijn het succespercentage, het aantal fouten per tijdseenheid, de gemiddelde tijd en de subjectieve tevredenheid van de gebruiker. Deze opdrachten zullen formatief verlopen. Er zal dus een moderator/observator aanwezig zijn die noteert hoe de gebruiker de opdrachten uitvoert. Na deze opdrachten zal de gebruiker nog een vragenlijst invullen om zo de SUS-score (System Usability Scale) te berekenen.

%---------- Verwachte resultaten ----------------------------------------------
\section{Verwachte resultaten}
\label{sec:verwachte_resultaten}

Ik verwacht dat de eindgebruiker er gemiddeld langer over zal doen om te wennen aan bepaalde functionaliteiten bij een gebrek aan in-app user training of onboarding. Ook verwacht ik dat een onboarding waarbij alle functionaliteiten aan bod komen een negatieve impact geeft in vergelijking met in-app user training per functionaliteit wanneer de gebruiker hiermee in contact komt.

%---------- Verwachte conclusies ----------------------------------------------
\section{Verwachte conclusies}
\label{sec:verwachte_conclusies}

Het nut van een goede onboarding zal bewezen worden. Alsook mag de onboarding enkel betrekking hebben op de meest essentiële functionaliteit.

In-app user training zal ervoor zorgen dat de eindgebruiker het gevoel heeft de functionaliteit onder de knie te hebben. Daardoor zal de gebruiker minder vlug afstand nemen van de applicatie. Ook zal het eenvoudiger zijn voor gebruikers met een beperkte technische achtergrond om gebruik te maken van de applicatie.



%%---------- Andere bijlagen --------------------------------------------------

\chapter{Interview In The Pocket}
\label{bijlage:interview-itp}

\section{Interview informatie}

Datum: 2 mei 2020. \\
Respondent: Quentin Braet (\href{https://www.inthepocket.com/}{In The Pocket})

\section[Vraag 1]{Hoe start je bij de ontwikkeling van een nieuwe applicatie aan de onboarding?}

\textit{Neem je de onboarding op in het designproces?}

Heel veel hangt af van het project. We gaan altijd eerst samen met de klant op zoek naar de requirements. Hiervoor starten we van de business requirements waaruit we zowel de  functionele als niet functionele requirements afleiden. Dit geldt ook voor de onboarding van een app. We vertrekken van de doelen (requirements) en kijken wat we nodig hebben om die te bereiken. Veel hangt af van hoe we gebruikers authenticeren en welke vereisten de klant daarbij heeft.

Bij de Payconiq by Bancontact app hebben we bijvoorbeeld heel zware security en legal constraints, maar ook complexiteit die geïntroduceerd wordt door de partners waarmee we samenwerken. Dat maakt dat we voor een heel grote uitdaging stonden om deze technisch zware onboarding flow toch behapbaar te houden voor de gebruiker.

Ook heb je bij mobile apps vaak meerdere permissies die je moet vragen, ook deze moeten worden meegenomen in het designproces. Je wil de gebruiker hier zoveel mogelijk begeleiden en informeren, maar het mag ook niet overweldigend zijn, teveel tekst op een scherm is vaak geen goed idee, de copy is vaak een belangrijk aandachtspunt.

Al deze requirements worden bij ons door de product manager en architect opgenomen in de intake gesprekken. Eens we een duidelijk beeld hiervan hebben, gaan we hiermee aan de slag om samen met een designer de eerste mockups te maken. Deze worden dan afgecheckt op allerlei vlakken, ook met de klant.

\section[Vraag 2]{Hoe test je het gehele onboarding proces binnen ITP?}

\textit{Test je direct bij echte gebruikers of is een rondje rond kantoor reeds voldoende?}

Onze mensen zijn getraind op interfaces ontwerpen, maar doordat wij zelf vaak power users zijn, is het niet altijd evident om te zien waar minder digitaal aangelegde gebruikers moeite mee gaan hebben. Dat is waarom wij vaak gaan ``user testen'', zeker voor grote projecten waar een foute implementatie van een complexe onboarding flow grote gevolgen kan hebben.

Bij user test leggen we de mockups voor aan mensen die we extern recruiten. Vaak een heel gevarieerd publiek, maar aan een 5 tot 10 tal mensen heb je vaak genoeg om de belangrijkste issues te spotten. Hierbij leggen we ze ofwel een printout van de mockups voor, maar vaak kijken we of we toch een clickable prototype kunnen maken om het geheel echter te laten aanvoelen. In deze sessies vergaar je vaak heel veel feedback over onlogische flows of copy en icoontjes die onduidelijk zijn.

Maar ook dit dekt niet alles: wanneer gebruikers langs komen voor zo een user test, dan zijn ze daar met een doel om al je opdrachten te voltooien. In het echt haken mensen gewoon af op het moment dat dingen onduidelijk zijn en hebben ze niet altijd de mogelijkheid om extra uitleg te vragen. Vandaar dat ook de data, vaak in de vorm van funnels, een grote rol spelen in het bijsturen van onboarding flows die live staan. Bovenop deze funnels kunnen ook A/B tests gemaakt worden om het effect van kleine aanpassingen te meten om zo tot een optimale flow te komen.

Naast user testing, doen we ook vaak beta testing, waarbij iedereen van het team (inclusief de klant), maar bij uitbreiding ook gans ITP in een beta programma zit voor onze apps. Dit zijn onze zogenaamde friendly users, die de productie app testen voor hij naar het brede publiek gaat. We krijgen nu eenmaal makkelijker feedback van onze collega's dan van mensen die we niet kennen. Eens de beta goedgekeurd is, gaan we over tot een phased rollout, wat betekent dat we geleidelijk aan meer productie gebruikers toegang geven tot de nieuwste versie. In deze periode houden we alle statistieken nog extra in de gaten.

\section[Vraag 3]{Welke elementen zijn zeker terug te vinden bij een goede onboarding volgens ITP?}

\textit{Denk aan: tooltips, welkomstbericht, ...}

De boodschap is altijd: keep it simple. Hoe korter de flow hoe beter. Toch is dit niet altijd mogelijk. Een aantal dingen die we zeker altijd doen:
\begin{itemize}
    \item Bij het vragen van permissies gaan we altijd eerst een scherm tonen die uitlegt waarom deze permissie gebruikt wordt.
    \item Vaak is er een legal scherm (terms \& conditions + privacy disclaimer), vraag de toestemmingen niet stiekem maar heel expliciet.
    \item Als je onboarding flow toch lang wordt, geef de gebruiker indicaties van hoe ver hij in de flow zit.
    \item Bij een lange onboarding flow: probeer eventueel de flow toch op te splitsen. Sommige functionaliteiten kan je misschien pas later activeren met een extra stukje onboarding achteraf. Dit helpt de gebruiker om snel een bruikbare app te zien.
    \item Introduceer geen nieuwe wachtwoorden als dat niet nodig is, gebruik indien mogelijk bestaande accounts.
    \item Een handleiding bij de app zou niet nodig mogen zijn, maar korte coachmarks kunnen wel helpen. Ideaal is ook om die niet allemaal tegelijk te geven bij het eerste gebruik, maar de gebruiker enkel het belangrijkste mee te geven in het begin, en meer geavanceerde features later pas te introduceren indien dat nog nodig is (wanneer je merkt dat een gebruiker die bijvoorbeeld nog niet gebruikt heeft).
    \item Maak het visueel, niemand leest graag veel tekst.
\end{itemize}

\section[Vraag 4]{Wanneer merk je dat een gebruiker moeilijkheden ondervindt bij een bepaalde functionaliteit?}

\textit{Zie je dit aan gedragingen van de gebruiker bij bijvoorbeeld een usability test?}

Ook hier hangt het af van welke phase we in zitten.

\textbf{Ontwerpfase}: de eerste stap is dat het team het zelf moeten begrijpen, dat klinkt logisch maar is soms al een enorme uitdaging (denk aan apps als Itsme \& Payconiq by Bancontact).

\textbf{User testing fase}: hier let je vooral op het gedrag van de testgebruikers en de vragen die ze stellen.

\textbf{Development \& QA fase}: ook hier komt vaak nog wat feedback van het team, door alles dieper te gaan bekijken komen soms nog rare kronkels naar boven of dingen die plots technisch moeilijker blijken dan gedacht.

\textbf{Beta fase}: de periode waarin we feedback krijgen van onze klant en andere collega's, als blijkt dat bepaalde zaken echt niet werken, kunnen die nog herwerkt worden voor we naar het brede publiek gaan.

\textbf{Phased rollout}: hier monitoren we allerhande statistieken en reviews. Zijn er flows die plots minder gebruikt worden? Komt dit door een technisch probleem? Vinden de mensen een nieuwe feature wel? Moeten we die duidelijker aangeven met een coachmark of een what's new? Indien nodig stoppen we de rollout en wordt het probleem eerst opgelost.

\textbf{Productie fase}: Hier spreekt de data van de onboardingsfunnel het meest, bij welk scherm of bij welke actie haken mensen af? Dit is natuurlijk enkel een indicator, hiermee weet je nog niet noodzakelijk de oorzaak.

\section[Vraag 5]{Hoe voorzie je een help sectie bij zeer complexe functionaliteiten?}

\textit{Volstaat een FAQ sectie? Neem je als het ware het handje van de gebruiker vast en doorloop je alle stappen in de applicatie?}

In principe vermijden we die koste wat het kost, help secties zijn vaak maar een lapmiddel om een minder goede UX toch bruikbaar te maken. In principe moet een UI zichzelf uitwijzen. Dit kan met een aantal kleine tricks zoals coachmarks op gepaste momenten (zie hierboven), een informatieve maar subtiele banner die tips \& tricks geeft... In sommige gevallen beperken we de info op het scherm zelf, waar goede verstaanders in principe genoeg mee hebben, maar zetten we toch ergens een link naar meer uitleg. Veel klanten zetten toch nog een FAQ op hun website en linken er naar via de app, al wordt die vaak aangevuld door informatie van hun first line support. Wij bekijken continu of we enkele van die opmerkingen niet beter kunnen verwerken in de UX van de app. Al hebben al die dingen natuurlijk wel een prijs, en moeten er trade offs gemaakt worden. In principe is alles oplosbaar, maar sommige technische implicaties zijn te kostelijk en die oplossen levert misschien ook maar een beperkte meerwaarde voor de gebruiker. Ook al vinden we dat soms zelf jammer omdat we weten dat het eigenlijk beter zou moeten kunnen, maar het is een constante evenwichtsoefening.


\chapter{Interview Cardify}
\label{bijlage:interview-cardify}

\section{Interview informatie}

Datum: 4 mei 2020. \\
Respondent: Seppe Vereecken (Product \& Growth bij \href{https://getcardify.com/}{Cardify})

\section[Vraag 1]{Hoe start je bij de ontwikkeling van een nieuwe applicatie aan de onboarding?}

\textit{Neem je de onboarding op in het designproces?}

Bij de ontwikkeling van Cardify start men nog niet met onboarding. Men start eerst met de implementatie van de effectieve functie en  achteraf bekijkt men waar en wanneer er onboarding zou voorzien moeten worden.

Omdat Cardify nog steeds een startup is leren we uit elke ervaring bij. Moest je me deze vraag binnen en jaar opnieuw tellen zal het antwoord waarschijnlijk volledig anders zijn.

Wat wel onmiddellijk geïmplementeerd werd bij onze app was een flow voor de registratie. Nadat de gebruiker alle gegevens hebben ingevuld wordt er een kaart getoond. Het is belangrijk dat de gebruiker het product zo snel mogelijk te zien krijgt.

\section[Vraag 2]{Hoe test je het gehele onboarding proces binnen Cardify?}

\textit{Test je direct bij echte gebruikers of is een rondje rond kantoor reeds voldoende?}

Wij gebruiken daar vooral ``superusers'' voor. Nu je het zo vraagt denk ik er aan dat we dat misschien toch iets officiëler moeten maken zodat we inderdaad een vaste groep gebruikers hebben die zich opgegeven hebben om nieuwe features te testen. In de Birdhouse (co-working kantoor) ging ik soms gewoon rond om features te laten testen door andere startups maar tot nu toe beperken we ons dus tot 'friends, family, fools' en superusers waarvan we weten dat ze engaged zijn.

Wanneer een gebruiker moeite ondervindt met bepaalde functies passen we de applicatie aan in de volgende iteratie. Dat kan gaan van kleine communicatieve aanpassingen, zoals popups of kleine UI veranderingen zoals kleuren of buttons maar dat kan ook als gevolg hebben dat we die feature volledig anders positioneren in de app.

Het uploaden van content was bijvoorbeeld een feature die in het begin minder belangrijk was voor ons en daardoor iets verder weg zat maar dus ook weinig gebruikt werd. We hebben dan een extra tussenscherm gemaakt alvorens ze contactgegevens kunnen aanpassen waardoor het uploaden van content een pak naar voor geschoven werd en nu dus ook meer gebruikt wordt.

Alles bij elkaar wordt er vaak nog ``op het gevoel'' gewerkt.

\section[Vraag 3]{Welke elementen zijn zeker terug te vinden bij een goede onboarding volgens Cardify?}

\textit{Denk aan: tooltips, welkomstbericht, ...}

Bij Cardify is het doel nog steeds om een product/dienst te verkopen. De klant maakt in de onboarding zo snel mogelijk kennis met de Cardify kaart. Deze kaart is na de onboarding ook meteen gepersonaliseerd, dit moet bij de klant een 'aha-moment' creëren. We laten de gebruiker ook kennis maken met alles wat mogelijk is, hierdoor leert de gebruiker meteen ook met deze functionaliteiten werken.

Bij Cardify willen we alles zo makkelijk mogelijk maken. Een goede onboarding bestaat dus uit veel visuals en weinig tekst. De gebruiker weet hoogstwaarschijnlijk graag hoever ze in de onboarding zitten, dus daarvoor tonen we voortgangsindicatoren.

\section[Vraag 4]{Wanneer merk je dat een gebruiker moeilijkheden ondervindt bij een bepaalde functionaliteit?}

\textit{Zie je dit aan gedragingen van de gebruiker bij bijvoorbeeld een usability test?}

Om initieel een onboarding te testen laten we een gebruiker de applicatie even gebruiken. Wanneer de gebruiker de applicatie sluit stellen we deze enkele vragen. Alsook vragen we achteraf om enkele taken uit te voeren. Wanneer deze mislukken zit er een fout in de onboarding, de gebruiker heeft niet goed geleerd met de app te werken.

In de app zitten ook enkele ``verplichte'' flows. Dit zijn flows die de gebruiker normaal gezien direct zou moeten doorlopen (zoals een kaart personaliseren en zijn/haar kaart laten scannen). Wanneer van zo’n flow afgeweken werd is er iets niet duidelijk voor de gebruiker en moeten we daarop anticiperen.

Op ons intern platform kunnen we beperkt monitoren welke functionaliteiten van de applicatie worden gebruikt. Wanneer een functionaliteit niet het gewenste resultaat behaalt spelen we ook hierop in. We plaatsen deze bijvoorbeeld voorop in de navigatie of we ``duwen'' de gebruiker al het ware naar de functionaliteit.

\section[Vraag 5]{Hoe voorzie je een help sectie bij zeer complexe functionaliteiten?}

\textit{Volstaat een FAQ sectie? Neem je als het ware het handje van de gebruiker vast en doorloop je alle stappen in de applicatie?}

Bij Cardify hebben we een beknopt help-center. Hier worden alle functionaliteiten grondig uitgelegd. Uit persoonlijke ervaring weten we wel dat zo’n help-center zelden gebruikt zal worden, daarom proberen we alles zo duidelijk mogelijk te maken in de app zelf.

In de web-platformen zitten vaak iets complexere stukken van het softwarepakket. Deze worden uitgelegd aan de hand van een tour. Deze tour loodst de gebruiker doorheen de belangrijkste flows door middel van actie-gedreven tooltips. Dit willen we echter niet verplichten.


\chapter{System Usability Scale template}
\label{bijlage:sus}

\includegraphics[page=1,width=.5\columnwidth]{../sus-template/sus-template.pdf}
\includegraphics[page=2,width=.5\columnwidth]{../sus-template/sus-template.pdf}

\chapter{Deelnameformulier}
\label{bijlage:deelnameformulier}

\section{Inleiding}

\textbf{Inschrijving applicatie testing}

Hey!

Bedankt voor je interesse in mijn bachelorproef. Voor mijn onderzoek moet ik enkele usability tests van software uitvoeren. In het kort wil dit dus zeggen dat ik een kleine mobiele applicatie loslaat op echt gebruikers! Na het invullen van deze enquête kom je te weten als je in aanmerking komt voor dit onderzoek.

Alvast bedankt!

Vanwege de aanhoudende Covid-19-situatie zal de usability test worden gehouden via een videogesprek.

\section{Mobiel toestel}

\subsection*{Type smartphone in bezit}

Gelieve "iPhone" te selecteren als je tijdelijk een iPhone kan lenen van een vriend, huisgenoot, ...

\begin{itemize}
    \item iPhone
    \item Android-toestel (Samsung, Oneplus, ...) \\ \textit{Bij keuze van deze optie wordt er direct doorverwezen naar sectie~\ref{sec:voldoet-niet}.}
    \item Anders \\ \textit{Bij keuze van deze optie wordt er direct doorverwezen naar sectie~\ref{sec:voldoet-niet}.}
\end{itemize}

\subsection*{Huidige iOS versie}

U vindt uw versie onder Instellingen > Algemeen > Info > Softwareversie

\begin{itemize}
    \item iOS 13.x.x
    \item iOS 12.x.x \\ \textit{Bij keuze van deze optie wordt er direct doorverwezen naar sectie~\ref{sec:voldoet-niet}.}
    \item Lager dan iOS 12.x.x \\ \textit{Bij keuze van deze optie wordt er direct doorverwezen naar sectie~\ref{sec:voldoet-niet}.}
\end{itemize}

\section{Over jezelf}

Jouw apparaat komt in aanmerking voor deze test! Gelieve enkele gegevens over jezelf achter te laten zodat ik jou kan contacteren.

Contactgegevens worden niet gedeeld.

\subsection*{Volledige naam}

Dit is een invulveld.

\subsection*{Leeftijdsgroep}

\begin{itemize}
    \item 70+
    \item 60-69
    \item 50-59
    \item 40-49
    \item 30-39
    \item 20-29
    \item 10-19
\end{itemize}

\subsection*{Geslacht}

\begin{itemize}
    \item Vrouw
    \item Man
    \item Non-binair
\end{itemize}

\subsection*{E-mailadres}

Dit is een invulveld.

\subsection*{Telefoonnummer}

Dit is een invulveld.

\subsection*{Ik heb een computer met een webcam en microfoon}

\begin{itemize}
    \item Ja
    \item Nee
\end{itemize}

\subsection*{Ik identificeer mezelf als technisch onderlegd}

Een smartphone heeft voor mij weinig tot geen geheimen

\begin{itemize}
    \item Ja
    \item Nee
\end{itemize}

\section{Oh nee...}
\label{sec:voldoet-niet}

\textit{Deze sectie wordt enkel getoond indien men niet voldoet aan bepaalde voorwaarden.}

Je komt helaas niet in aanmerking voor deze test. De dummy-applicatie werd geschreven voor iOS-apparaten met versie 13 en hoger. Jouw apparaat voldoet niet aan deze voorwaarden.

\section{Afsluiting}

Bedankt voor de interesse. Indien je jouw contactgegevens hebt moeten achterlaten kom jij in aanmerking voor deze test. In dat geval probeer ik zo snel mogelijk contact met je op te nemen. Vergeet ook niet dit formulier om te delen!

Blijf in uw kot en was uw handen. 

Jakob Lierman


\chapter{Berekeningen R onderzoeksvraag 1}
\label{bijlage:r-1}

\lstinputlisting{scripts/test_rq1.R}

\chapter{Berekeningen R onderzoeksvraag 2}
\label{bijlage:r-2}

\lstinputlisting{scripts/test_rq2.R}

\chapter{Berekeningen R onderzoeksvraag 3}
\label{bijlage:r-3}

\lstinputlisting{scripts/test_rq3.R}

\chapter{Berekeningen R onderzoeksvraag 4}
\label{bijlage:r-4}

\lstinputlisting{scripts/test_rq4.R}

%%---------- Referentielijst --------------------------------------------------

\printbibliography[heading=bibintoc]

\end{document}
