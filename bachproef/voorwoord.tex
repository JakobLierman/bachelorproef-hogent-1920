%%=============================================================================
%% Voorwoord
%%=============================================================================

\chapter*{\IfLanguageName{dutch}{Woord vooraf}{Preface}}
\label{ch:voorwoord}

Deze bachelorproef werd geschreven in het kader van het behalen van het diploma ``Bachelor in de Toegepaste Informatica'', afstudeerrichting Mobile Apps.

Mijn interesses gaan verder dan enkel de technische kant van mijn opleiding. Net daarom heb ik voor dit onderwerp gekozen. Het verkennen van verschillende \acrshort{acr:ux} en \acrshort{acr:ui}-elementen sprak me enorm aan. Het was een leerrijke ervaring om applicaties en IT voor te leggen aan iedereen in mijn omgeving, jong en oud, om dan van dichtbij te observeren hoe men met de technologie omspringt.

Deze scriptie zou niet tot stand zijn gekomen zonder enkele personen. Hierbij wil ik van dit voorwoord graag gebruik maken om deze personen te bedanken. 

Zonder Bert Maurau, mijn co-promotor, had ik deze scriptie niet kunnen aanvullen met gegevens en voorbeelden vanuit Cardify. Hij en Seppe Vereecken stonden steeds klaar om mijn vragen te beantwoorden en deze scriptie aan te vullen met nuttige info en voorbeelden.

Deze scriptie werd geschreven tijdens de Covid-19-crisis. Bram Van de velde heeft tools aangeboden om mijn proef zo vlot mogelijk te laten verlopen vanop afstand. Zo kon ik de proof-of-concept applicatie delen met iPhone-gebruikers zonder hiervoor met de participant in contact te komen.

Quentin Braet gaf deze scriptie een extra dimensie door zijn kennis van bij In The Pocket te delen.

Karine Samyn, mijn promotor, volgde mijn voortgang nauw op en gaf op regelmatige basis feedback. Door deze opbouwende kritiek werd deze scriptie zonder veel problemen tot een goed einde gebracht.

Amber Priem hielp me doorheen de statistische analyses, waardoor ik mijn gemeten waardes kon omzetten in een conclusie. Ook was ze steeds de eerste die klaar stond om een tekstblok na te lezen of te controleren op grammaticale fouten.

Ten slotte had ik graag nog mijn ouders bedankt voor de financiële steun om deze opleiding tot een goed einde te brengen. Uiteraard had ik hen en iedereen die mijn scriptie even doornam graag nogmaals bedankt om deze tekst na te lezen, ook al ligt deze toch wel uit hun interessegebied.

Ik wens u veel leesplezier toe.

Jakob Lierman

Gent, \shortdate\today


