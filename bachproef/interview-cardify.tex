\section{Interview informatie}

Datum: 4 mei 2020. \\
Respondent: Seppe Vereecken (Product \& Growth bij \href{https://getcardify.com/}{Cardify})

\section[Vraag 1]{Hoe start je bij de ontwikkeling van een nieuwe applicatie aan de onboarding?}

\textit{Neem je de onboarding op in het designproces?}

Bij de ontwikkeling van Cardify start men nog niet met onboarding. Men start eerst met de implementatie van de effectieve functie en  achteraf bekijkt men waar en wanneer er onboarding zou voorzien moeten worden.

Omdat Cardify nog steeds een startup is leren we uit elke ervaring bij. Moest je me deze vraag binnen en jaar opnieuw tellen zal het antwoord waarschijnlijk volledig anders zijn.

Wat wel onmiddellijk geïmplementeerd werd bij onze app was een flow voor de registratie. Nadat de gebruiker alle gegevens hebben ingevuld wordt er een kaart getoond. Het is belangrijk dat de gebruiker het product zo snel mogelijk te zien krijgt.

\section[Vraag 2]{Hoe test je het gehele onboarding proces binnen Cardify?}

\textit{Test je direct bij echte gebruikers of is een rondje rond kantoor reeds voldoende?}

Wij gebruiken daar vooral ``superusers'' voor. Nu je het zo vraagt denk ik er aan dat we dat misschien toch iets officiëler moeten maken zodat we inderdaad een vaste groep gebruikers hebben die zich opgegeven hebben om nieuwe features te testen. In de Birdhouse (co-working kantoor) ging ik soms gewoon rond om features te laten testen door andere startups maar tot nu toe beperken we ons dus tot 'friends, family, fools' en superusers waarvan we weten dat ze engaged zijn.

Wanneer een gebruiker moeite ondervindt met bepaalde functies passen we de applicatie aan in de volgende iteratie. Dat kan gaan van kleine communicatieve aanpassingen, zoals popups of kleine UI veranderingen zoals kleuren of buttons maar dat kan ook als gevolg hebben dat we die feature volledig anders positioneren in de app.

Het uploaden van content was bijvoorbeeld een feature die in het begin minder belangrijk was voor ons en daardoor iets verder weg zat maar dus ook weinig gebruikt werd. We hebben dan een extra tussenscherm gemaakt alvorens ze contactgegevens kunnen aanpassen waardoor het uploaden van content een pak naar voor geschoven werd en nu dus ook meer gebruikt wordt.

Alles bij elkaar wordt er vaak nog ``op het gevoel'' gewerkt.

\section[Vraag 3]{Welke elementen zijn zeker terug te vinden bij een goede onboarding volgens Cardify?}

\textit{Denk aan: tooltips, welkomstbericht, ...}

Bij Cardify is het doel nog steeds om een product/dienst te verkopen. De klant maakt in de onboarding zo snel mogelijk kennis met de Cardify kaart. Deze kaart is na de onboarding ook meteen gepersonaliseerd, dit moet bij de klant een 'aha-moment' creëren. We laten de gebruiker ook kennis maken met alles wat mogelijk is, hierdoor leert de gebruiker meteen ook met deze functionaliteiten werken.

Bij Cardify willen we alles zo makkelijk mogelijk maken. Een goede onboarding bestaat dus uit veel visuals en weinig tekst. De gebruiker weet hoogstwaarschijnlijk graag hoever ze in de onboarding zitten, dus daarvoor tonen we voortgangsindicatoren.

\section[Vraag 4]{Wanneer merk je dat een gebruiker moeilijkheden ondervindt bij een bepaalde functionaliteit?}

\textit{Zie je dit aan gedragingen van de gebruiker bij bijvoorbeeld een usability test?}

Om initieel een onboarding te testen laten we een gebruiker de applicatie even gebruiken. Wanneer de gebruiker de applicatie sluit stellen we deze enkele vragen. Alsook vragen we achteraf om enkele taken uit te voeren. Wanneer deze mislukken zit er een fout in de onboarding, de gebruiker heeft niet goed geleerd met de app te werken.

In de app zitten ook enkele ``verplichte'' flows. Dit zijn flows die de gebruiker normaal gezien direct zou moeten doorlopen (zoals een kaart personaliseren en zijn/haar kaart laten scannen). Wanneer van zo’n flow afgeweken werd is er iets niet duidelijk voor de gebruiker en moeten we daarop anticiperen.

Op ons intern platform kunnen we beperkt monitoren welke functionaliteiten van de applicatie worden gebruikt. Wanneer een functionaliteit niet het gewenste resultaat behaalt spelen we ook hierop in. We plaatsen deze bijvoorbeeld voorop in de navigatie of we ``duwen'' de gebruiker al het ware naar de functionaliteit.

\section[Vraag 5]{Hoe voorzie je een help sectie bij zeer complexe functionaliteiten?}

\textit{Volstaat een FAQ sectie? Neem je als het ware het handje van de gebruiker vast en doorloop je alle stappen in de applicatie?}

Bij Cardify hebben we een beknopt help-center. Hier worden alle functionaliteiten grondig uitgelegd. Uit persoonlijke ervaring weten we wel dat zo’n help-center zelden gebruikt zal worden, daarom proberen we alles zo duidelijk mogelijk te maken in de app zelf.

In de web-platformen zitten vaak iets complexere stukken van het softwarepakket. Deze worden uitgelegd aan de hand van een tour. Deze tour loodst de gebruiker doorheen de belangrijkste flows door middel van actie-gedreven tooltips. Dit willen we echter niet verplichten.
