\chapter{\IfLanguageName{dutch}{Methodologie}{Methodology}}
\label{ch:methodologie}

% TODO: Hoe ben je te werk gegaan? Verdeel je onderzoek in grote fasen, en licht in elke fase toe welke stappen je gevolgd hebt. Verantwoord waarom je op deze manier te werk gegaan bent. Je moet kunnen aantonen dat je de best mogelijke manier toegepast hebt om een antwoord te vinden op de onderzoeksvraag.

In dit hoofdstuk zal besproken worden hoe het experiment in zijn werk gaat. % TODO

\section{Het experiment}
\label{sec:experiment}

Om te bekijken indien een implementatie van bepaalde learnability technieken invloed heeft op de eindgebruiker zullen er usability tests worden uitgevoerd bij testpersonen. Welke deze testpersonen zijn en hoe er deelnemers werden verzameld is te raadplegen in hoofdstuk~\ref{sec:deelnemers}.

Bij dit experiment zal er vooral gefocust worden op laboratory testing (zie hoofdstuk~\ref{sec:usability-testing:lab-field-testing}). Dit wil zeggen dat de testpersoon de applicatie zal testen in het bijzijn van een moderator, maar niet strikt noodzakelijk in de omgeving waarin de applicatie in een reëel scenario gebruikt zou worden. De moderator noteert hierbij alle bevindingen van de testpersoon, alsook waar deze eventueel moeilijkheden ondervindt.

De test zal opgesplitst worden in twee delen. De ene groep zal starten met een usability test van een proof-of-concept applicatie waarin alle learnability technieken in verwerkt zijn, gevolgd door een vragenlijst. Daarna krijgen zij de applicatie zonder deze technieken en wordt van de deelnemers verwacht dat ze hun mening geven over beide applicaties. De andere groep doet het omgekeerde; deze starten met een applicatie zonder de learnability technieken en mogen daarna de applicatie met deze technieken ontdekken. Tijdens deze usability tests wordt van de testpersonen verwacht dat deze een reeks taken tot een goed einde proberen te brengen. De tijd die nodig is om deze taak uit te voeren zal opgenomen worden, alsook zal er genoteerd worden waar er moeilijkheden ontstaan. Zo is het mogelijk om te meten hoeveel sneller een bepaalde taak beëindigd werd bij een de verschillende applicaties. Men zal dezelfde opdrachten meermaals uitvoeren, om zo te testen indien men effectief onthouden heeft wat men voordien uitvoerde.



\section{De deelnemers}
\label{sec:deelnemers}

Aan dit experiment kan quasi iedereen deelnemen indien deze een basiskennis hebben van een smartphone. Leeftijd, afkomst, achtergrond, kennis en andere factoren zijn van weinig belang. Er moet echter wel voldoende variatie zijn. Wanneer de groep deelnemers enkel uit jonge personen met een technische achtergrond bestaat zal het resultaat sterk beïnvloed worden. Het is dus de bedoeling om zowel jong, oud, technisch onderlegd en technisch leek op te nemen in de poule van deelnemers.

Deelnemers worden gezocht zowel in de kring van familie en vrienden als daarbuiten. De resultaten van de laboratory tests met deze deelnemers gingen normaal aangevuld worden met resultaten uit guerilla tests (zie hoofdstuk~\ref{sec:usability-testing:testmethoden:guerilla}). Deze scriptie werd opgesteld tijdens de Covid-19-situatie en het was dus nie mogelijk en/of veilig om guerilla testing uit te voeren.

Omdat uit veiligheidsoverwegingen geopteerd werd om de test op afstand uit te voeren (de moderator is dus niet fysiek aanwezig bij de testpersoon) moet de testpersoon de proof-of-concept applicatie installeren op hun eigen toestel. De applicatie is geschreven voor het mobiele besturingssysteem van Apple (iOS 13 of hoger) en werd geoptimaliseerd voor gebruik op een smartphone (iPhone of iPod touch). Bij het verzamelen van gegevens van potentiële deelnemers werd er gevraagd indien deze in het bezit zijn van een toestel dat aan deze voorwaarden voldoet.

Een deelnameformulier werd via verschillende kanalen (sociale netwerken, mond-tot-mondreclame, ...) verspreid om zoveel mogelijk potentiële deelnemers te bereiken. Het deelnameformulier werd opgesteld in \href{https://forms.office.com/}{Microsoft Forms}. Het volledige formulier is bijgevoegd in bijlage~\ref{bijlage:deelnameformulier}.

% TODO - X vervangen
In totaal werd het formulier x keer ingevuld. Daaruit werden x personen geselecteerd om deel te nemen aan de test. In figuur~X is te zien in welke leeftijdsgroep de deelnemers zich bevinden, welk geslacht deze hebben en indien ze zichzelf al dan niet zien als technisch onderlegd.
