%%=============================================================================
%% Conclusie
%%=============================================================================

\chapter{Conclusie}
\label{ch:conclusie}

% TODO: Trek een duidelijke conclusie, in de vorm van een antwoord op de onderzoeksvra(a)g(en). Wat was jouw bijdrage aan het onderzoeksdomein en hoe biedt dit meerwaarde aan het vakgebied/doelgroep? Reflecteer kritisch over het resultaat. In Engelse teksten wordt deze sectie ``Discussion'' genoemd. Had je deze uitkomst verwacht? Zijn er zaken die nog niet duidelijk zijn? Heeft het onderzoek geleid tot nieuwe vragen die uitnodigen tot verder onderzoek?

In dit onderzoek werd een antwoord gegeven op de onderzoeksvraag ``Kan een (betere) onboarding en in-app user training ervoor zorgen dat de eindgebruiker een beter inzicht heeft op de totale functionaliteit van een grote applicatie?''. Hiervoor werd een proof-of-concept applicatie opgesteld met en zonder learnability-elementen. Deze werd getest door twee groepen participanten.

De resultaten van deze proef gaven aan dat de learnability-elementen zeker hun nut hebben. Echter moet men zeer voorzichtig omspringen met waar en wanneer men deze plaatst. Zoals vooraf werd besproken is het belangrijk om zich te beperken tot de essentie, een gebruiker leert de details van de applicatie pas achteraf kennen. Zo wordt de gebruiker niet meteen overspoelt met informatie. Frequent de applicatie laten testen om zo te bekijken waar er hulp moet voorzien worden, is één van de belangrijkste stappen bij het implementeren van learnability in software.

Er bestaan verschillende technieken om onboarding op een correcte manier te implementeren. Niet elke techniek kan voor elke applicatie of use case gebruikt worden. De ontwikkelaar van de software bekijkt best welke technieken geschikt zijn voor zijn of haar specifieke use case. Deze verschillen in \acrshort{acr:ux} voor verschillende applicaties kunnen eventueel onderzocht worden in een verder onderzoek.

De verspreiding van help-elementen doorheen de applicatie is een volgend belangrijk aspect bij de implementatie van onboarding en help-elementen. Door de gebruiker niet initieel te overspoelen met info, maar deze pas geleidelijk aan nieuwe functionaliteiten aan te leren zal de gebruiker de werking van deze functionaliteiten beter onthouden. De gebruiker zal ook minder snel de aandacht verliezen dan tijdens bijvoorbeeld één grote rondleiding.

Er werd voor deze applicatie geen significante relatie gevonden tussen (het gebrek aan) in-app user training en de gebruiksduur en/of levensduur van de applicatie. Wanneer een bedrijf hun software test bij gebruikers die ook effectief een meerwaarde hebben aan de functionaliteiten van deze software kan dit resultaat met hoogste waarschijnlijkheid variëren. Dit kan eventueel bewezen worden in een verder onderzoek.

Een interessante ondervinding uit deze scriptie is het feit dat het mogelijk is de gebruiker aan te zetten tot het gebruik van bepaalde functionaliteiten. Door deze functionaliteiten aan te halen in de onboarding is de gebruiker meer geneigd deze daar uit te proberen en verder te gebruiken. Dit kan interessant zijn om bijvoorbeeld accountcreatie in de kijker te zetten.
