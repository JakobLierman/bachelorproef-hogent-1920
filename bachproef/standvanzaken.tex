\chapter{\IfLanguageName{dutch}{Stand van zaken}{State of the art}}
\label{ch:stand-van-zaken}

% Tip: Begin elk hoofdstuk met een paragraaf inleiding die beschrijft hoe dit hoofdstuk past binnen het geheel van de bachelorproef. Geef in het bijzonder aan wat de link is met het vorige en volgende hoofdstuk.

Zoals uit vorig hoofdstuk kan worden afgeleid zal deze scriptie onderzoek doen naar de implementatie en effecten van bepaalde UX en UI-elementen. Maar alvorens van start te gaan hiermee zal eerst gekeken worden naar wat de algemene rol is van UX en UI in software-ontwikkeling.

\section{User experience in software}
\label{sec:user-experience-in-software}

In het traditioneel proces van het ontwikkelen van software staat de functionaliteit centraal. Ontwikkelaars bekijken alle vereisten en starten met de belangrijkste. Functionaliteit krijgt hier doorgaans de voorkeur. \textcite{Harutyunyan2019} stelden vast dat dit de laatste jaren echter aan het wijzigen is. De traditionele softwareontwikkeling is plaats aan het maken voor softwareontwikkeling met user experience in het achterhoofd. Dit fenomeen noemt men User Experience Design of kortweg UXD. Omdat de term UXD in de literatuur nog sterk evolueert heeft dit nog geen algemeen aanvaarde definitie. Men kan stellen dat user experience design een proces is waarbij men gebruiksgedrag zal manipuleren aan de hand van de bruikbaarheid en wenselijkheid in de interactie met een product. 

Men doet al lang onderzoek naar user experience in software. Zo toonden \textcite{Carroll1984} het belang van training in complexe systemen al aan anno 1984. Deze training van gebruikers behandelen we later in dit hoofdstuk.

\begin{figure}[h]
    \centering
    \def\svgwidth{.8\columnwidth}
    \input{./img/user-experience-waarom-wat-hoe.pdf_tex}
    \caption{Het waarom, wat en hoe van user experience design}
    \label{fig:ux-waarom-wat-hoe}
\end{figure}

Een UX-designer bekijkt het product niet enkel als een het product zelf. Deze persoon analyseert hoe de eindgebruiker het product in gebruik neemt en past het product aan zodat de gebruikservaring optimaal is. De designer neemt het \textit{waarom}, \textit{wat} en \textit{hoe} van productgebruik in acht (zie figuur~\ref{fig:ux-waarom-wat-hoe}) \autocite{Hassenzahl2013}. De \textit{wat} in productgebruik verwijst gewoonlijk naar wat een gebruiker kan doen door middel van het product. Dit is bijvoorbeeld ``een foto maken'' of ``een spel kopen''. De \textit{hoe} staat dan ook effectief voor de gebruiker het product gebruikt. Dit is meer op een operationeel niveau zoals het navigeren door software met behulp van knoppen en andere attributen. De designer zal zich voornamelijk focussen op het ``hoe'' van het productgebruik. Dit omvat de gegeven functionaliteit op een aantrekkelijke manier zeer toegankelijk maken.


% TODO
