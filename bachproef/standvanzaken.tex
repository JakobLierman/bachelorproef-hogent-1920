\chapter{\IfLanguageName{dutch}{Stand van zaken}{State of the art}}
\label{ch:stand-van-zaken}

% Tip: Begin elk hoofdstuk met een paragraaf inleiding die beschrijft hoe dit hoofdstuk past binnen het geheel van de bachelorproef. Geef in het bijzonder aan wat de link is met het vorige en volgende hoofdstuk.

Zoals uit vorig hoofdstuk kan worden afgeleid zal deze scriptie onderzoek doen naar de implementatie en effecten van bepaalde UX en UI-elementen. Maar alvorens van start te gaan hiermee zal eerst gekeken worden naar wat de algemene rol is van UX en UI in software-ontwikkeling.

\section{User experience in software}
\label{sec:user-experience-in-software}

In het traditioneel proces van het ontwikkelen van software staat de functionaliteit centraal. Ontwikkelaars bekijken alle vereisten en starten met de belangrijkste. Functionaliteit krijgt hier doorgaans de voorkeur. \textcite{Harutyunyan2019} stelden vast dat dit de laatste jaren echter aan het wijzigen is. De traditionele softwareontwikkeling is plaats aan het maken voor softwareontwikkeling met user experience in het achterhoofd. Dit fenomeen noemt men User Experience Design of kortweg UXD. Omdat de term UXD in de literatuur nog sterk evolueert heeft dit nog geen algemeen aanvaarde definitie. Men kan stellen dat user experience design een proces is waarbij men gebruiksgedrag zal manipuleren aan de hand van de bruikbaarheid en wenselijkheid in de interactie met een product. 

Men doet al lang onderzoek naar user experience in software. Zo toonden \textcite{Carroll1984} het belang van training in complexe systemen al aan anno 1984. Deze training van gebruikers behandelen we later in dit hoofdstuk.

\begin{figure}[h]
    \centering
    \def\svgwidth{.8\columnwidth}
    \input{./img/user-experience-waarom-wat-hoe.pdf_tex}
    \caption{Het waarom, wat en hoe van user experience design}
    \label{fig:ux-waarom-wat-hoe}
\end{figure}

Een UX-designer bekijkt het product niet enkel als een het product zelf. Deze persoon analyseert hoe de eindgebruiker het product in gebruik neemt en past het product aan zodat de gebruikservaring optimaal is. De designer neemt het \textit{waarom}, \textit{wat} en \textit{hoe} van productgebruik in acht (zie figuur~\ref{fig:ux-waarom-wat-hoe}) \autocite{Hassenzahl2013}. De \textit{wat} in productgebruik verwijst gewoonlijk naar wat een gebruiker kan doen door middel van het product. Dit is bijvoorbeeld ``een foto maken'' of ``een spel kopen''. De \textit{hoe} staat dan ook effectief voor de gebruiker het product gebruikt. Dit is meer op een operationeel niveau zoals het navigeren door software met behulp van knoppen en andere attributen. De designer zal zich voornamelijk focussen op het ``hoe'' van het productgebruik. Dit omvat de gegeven functionaliteit op een aantrekkelijke manier zeer toegankelijk maken.

\section{Belangrijke factoren bij user experience}
\label{sec:belangrijke-factoren-bij-user-experience}

Er zijn talloze factoren die ervoor zorgen dat men op een verschillende manier naar dezelfde applicatie moet kijken. Zo zijn de gebruikers allemaal verschillend, maar er moet ook rekening gehouden worden met verschillende omgevingen, culturen, enz. Een toestel om parameters op te meten bij schepen moet dus waterbestendig zijn. Een kindvriendelijke tablet is best schokbestendig. Een applicatie om notities te maken bij vergaderingen is best geluidsloos.

De factoren kunnen gegroepeerd worden in vijf groepen (zie figuur~\ref{fig:ux-factoren}). Culturele factoren omvatten bijvoorbeeld religie, taal en gewoontes. Zo moet bij het ontwerpen van een website gericht op asielzoekers bijvoorbeeld rekening houden met het gebrek aan kennis van de landstaal.

Een mobiele applicatie waarbij de gebruiker een vervoersbewijs moet voorleggen op een voertuig van het openbaar vervoer zal bijvoorbeeld rekening moeten houden met het feit dat de sociale factor tijdsdruk hier belangrijk is. Indien deze applicatie niet tijdig het vervoersbewijs laat zien zal er een hele wachtrij ontstaan die dan vertragingen tot gevolg heeft.

Je kan factoren met betrekking tot de context waarin het product gebruikt wordt opvatten als bijvoorbeeld tijd en locatie. Bij het bestellen van een pakket krijg je vaak een tracking-link waarbij ook het tijdstip van levering staat. Een internationale leverancier moet dus zeker voorzien dat de tijd van de levering in de juiste tijdzone weergegeven wordt.

De gebruiker zelf verschilt uiteraard ook. Een applicatie gericht op een ouder publiek voorziet best grote tekst en duidelijke iconen.

Het product zelf moet uiteindelijk ook nog bruikbaar zijn en alle functionaliteiten moeten eenvoudig bereikbaar zijn.
Een hele boterham voor de user experience designer om onderzoek naar te doen voor zijn use case.

\begin{figure}
    \centering
    \def\svgwidth{.8\columnwidth}
    \input{./img/user-experience-factoren.pdf_tex}
    \caption{Belangrijke factoren bij user experience}
    \label{fig:ux-factoren}
\end{figure}

\textcite{Morville2004} verdeelde user experience op een andere manier. Hij maakt gebruik van de user experience honingraat (zie figuur~\ref{fig:ux-facets}) die user experience opsplitst in zeven onderdelen.

% TODO - https://www.interaction-design.org/literature/article/the-7-factors-that-influence-user-experience

\begin{itemize}
    \item \textbf{Nuttig.}
    Alle producten moeten een zeker nut hebben. Een applicatie mag niet zomaar een tool zijn van het management maar moet een zekere waarde hebben voor de eindgebruiker.
    \item \textbf{Bruikbaar.}
    De bruikbaarheid of usability van een product is een van de belangrijkste kenmerken van de user experience. Het is echter niet het enige kenmerk. Bruikbaarheid en gebruiksgemak zijn dus essentieel maar niet voldoende.
    \item \textbf{Gewenst.}
    De zoektocht naar een efficiënte applicatie mag de branding, het image en de esthetiek van de applicatie niet achterwege laten. Hoe wenselijker het product is, hoe meer de gebruiker erover zal opscheppen tegen potentieel nieuwe gebruikers.
    \item \textbf{Vindbaar.}
    Software moet eenvoudig te navigeren zijn. Gebruikers moeten vlot kunnen vinden wat ze nodig hebben.
    \item \textbf{Toegankelijk.}
    Software moet toegankelijk zijn voor alle doelgroepen. Een gebruiker met een handicap mag geen hindernissen ondervinden bij het gebruik ervan.
    \item \textbf{Geloofwaardig.}
    De design elementen gebruikt in de software moeten ervoor zorgen dat de gebruikers vertrouwen hebben in de informatie die we hen meedelen.
    \item \textbf{Waardevol.}
    De software moet waarde leveren voor de organisatie. De organisatie zal er naar streven dat de winst en klanttevredenheid sterk toenemen.
\end{itemize}

\begin{figure}
    \centering
    \def\svgwidth{.8\columnwidth}
    \input{./img/user-experience-facets.pdf_tex}
    \caption{De user experience honingraat}
    \label{fig:ux-facets}
\end{figure}

User experience design is een zeer creatief concept. Door deze creativiteit zijn er uiteraard verschillende meningen over hoe men user experience moet definiëren, omschrijven en indelen. Mits de belangrijkste vermeld zijn gaan we hier niet verder op in.

\subsection{User experience design in de praktijk}\label{sec:user-experience-in-software:user-experience-design-in-de-praktijk}

% https://medium.com/@jtnakagawa/nothing-left-to-take-away-437eb23c2ae8
Een eenvoudige applicatie moet simpel in gebruik zijn, dat is waar \textbf{Medium} op inzet. Medium is een online platform voor schrijvers en lezers. Bij een blog-artikel moet men focussen op de inhoud van het artikel. Door een gebrek aan kleurgebruik en een goede keuze van het lettertype is Medium gebruiksvriendelijker dan de papieren krant. Afbeeldingen zijn groot en duidelijk, de titel springt eruit en op enkele iconen na zijn er weinig tot geen afleidingen te bespeuren (zie figuur~\ref{fig:ux-voorbeeld-medium:desktop}). Medium trekt deze lijn door naar hun mobiele applicatie (zie figuur~\ref{fig:ux-voorbeeld-medium:mobiel}). Hier implementeerde men ook een donkere variant. Deze variant zorgt voor leescomfort in donkere omgevingen en in sommige gevallen ook voor batterijbesparing \autocite{Jin2017}.

\begin{figure}
    \centering
    \includegraphics[width=.8\columnwidth]{voorbeeld-medium-desktop}
    \caption{Artikel op Medium weergegeven in een desktop-omgeving}
    \label{fig:ux-voorbeeld-medium:desktop}
\end{figure}

\begin{figure}
    \centering
    \subfloat[Donkere gebruikersomgeving]{{\includegraphics[width=.4\columnwidth]{voorbeeld-medium-mobiel-donker}}}
    \qquad
    \subfloat[Lichte gebruikersomgeving]{{\includegraphics[width=.4\columnwidth]{voorbeeld-medium-mobiel-licht}}}
    \caption{Artikel op Medium weergegeven in een mobiele omgeving}
    \label{fig:ux-voorbeeld-medium:mobiel}
\end{figure}

% TODO
