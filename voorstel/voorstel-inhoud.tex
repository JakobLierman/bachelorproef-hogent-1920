%---------- Inleiding ---------------------------------------------------------

\section{Introductie} % The \section*{} command stops section numbering
\label{sec:introductie}

Technologie neemt overal de overhand. Meer leeftijdsgroepen maken gebruiken van web- en mobiele applicaties. Deze softwarepakketten worden voor alle mogelijke doeleinden gebruikt, denk maar aan een management platform in een bedrijf, of een mobiel spel om tegen je vrienden te quizzen.

Deze applicaties moeten op een bepaalde manier aan de eindgebruiker uitgelegd worden. Er bestaan verscheidene technieken om alle functionaliteiten binnen de software aan de man te brengen. Boeken deze technieken wel het gewenste resultaat? Is de ene techniek beter dan de andere?

In deze bachelorproef onderzoeken we de onboarding en in-app user training aan de hand van volgende onderzoeksvragen en deelvragen:

\begin{itemize}
    \item Kan een (betere) onboarding en in-app user training ervoor zorgen dat de eindgebruiker een beter inzicht heeft op de totale functionaliteit van een grote applicatie?
    \item Hoe een grote hoeveelheid aan functionaliteiten beheersbaar houden voor de eindgebruiker?
    \item Heeft (het gebrek aan) in-app user training effect op de gebruiksduur en/of levensduur van de applicatie?
    \item Hoe de eindgebruiker wegwijs maken in een grote applicatie?
\end{itemize}

%---------- Stand van zaken ---------------------------------------------------

\section{Literatuurstudie}
\label{sec:state-of-the-art}

\subsection{Wat is 'onboarding'?}
Er bestaan al verscheidene technieken die mogelijks geïmplementeerd kunnen worden in een applicatie om het de gebruiker eenvoudiger te maken deze applicatie te gebruiken. Eén van de bekendste technieken om een gebruiker wegwijs te maken binnen een softwarepakket is 'onboarding'. Onboarding vindt plaats wanneer u de software voor de eerste maal opstart. Het geeft de eindgebruiker een duidelijk inzicht van hoe de applicatie werkt, hoe de functionaliteiten in zijn werk gaan en welke voordelen er worden aangeboden~\autocite{Cardoso2017}.

\subsection{Voorgaand onderzoek}
Bestaand onderzoek bestudeert een manier van onboarding en in-app user training door middel van een 'Learning Center'~\autocite{CamachoHerrero2019}. Hierbij maakt men gebruik van een in-app onderwijsfunctie die de eindgebruiker alle mogelijke functionaliteiten van de software kan uitleggen. Dit platform wil de gebruiker meer betrekken en aanmoedigen om alle elementen te voltooien door middel van gamification elementen zoals checklists of voortgangsbalken. Per onderdeel zelf werd grotendeels gebruik gemaakt van video's die alle uitleg voorzien. In deze bachelorproef zal verder onderzocht worden als het voordeliger is om dit platform op te splitsen in kleine stukken per functionaliteit. Pas wanneer de gebruiker in contact komt met bepaalde functionaliteiten zal de hij de in-app training te zien krijgen.

Onboarding is vaak ook meer dan enkel uitleg over de applicatie. Ook elementen zoals de registratie zijn een stap in het proces om met de software overweg te kunnen~\autocite{renz2014}. Ook een welkomstmail is een veelgebruikt middel om de gebruiker vertrouwd te maken met de applicatie.

Onboarding is verschillend voor webapplicaties als mobiele applicaties~\autocite{RamirezAlvarez2018}. In beide gevallen moet grondig onderzocht worden wat de gebruiker wil bereiken en het gedrag van deze gebruikersgroep is. Applicaties gericht op een jeugdig publiek zullen de onboarding anders aanpakken dan softwarepakketten gericht op het vervullen van een taak in een bedrijfsomgeving.

\subsection{UX testing}
Voorgaand onderzoek maakt duidelijk dat er veel voorbereiding nodig is op vlak van user experience (UX) en usability testing. Om te weten wat de eindgebruiker wil bereiken en welke acties die daarvoor zal ondernemen is er genoeg inlevingsvermogen nodig~\autocite{gualtieri2009}.
Ook mag men er nooit vanuit gaan dat de ontwikkelaar het bij het rechte eind heeft. Hij is immers niet de eindgebruiker en bekijkt de probleemstelling op een andere manier.

%---------- Methodologie ------------------------------------------------------
\section{Methodologie}
\label{sec:methodologie}

Het onderzoeken van technieken zoals in-app user training en bepaalde manieren van het implementeren van deze technieken zal vooral gebeuren aan de hand van experimenten in het kader van usability testing.

Hierbij zal gebruik gemaakt worden van enkele varianten van dummy-applicaties (hetzij met of zonder gebruik van bovenstaande technieken). Gebruikers van de dummy-applicatie zullen fysiek de software doorlopen en enkele opdrachten voltooien. Aan de hand van metingen op basis van een aantal criteria zal kunnen worden besloten als er over een succesvolle test mag gesproken worden. Deze criteria zijn onder meer, maar niet beperkt tot; het succespercentage, het aantal fouten per tijdseenheid, de gemiddelde tijd en de subjectieve tevredenheid van de gebruiker.

%---------- Verwachte resultaten ----------------------------------------------
\section{Verwachte resultaten}
\label{sec:verwachte_resultaten}

Ik verwacht dat de eindgebruiker er gemiddeld langer over zal doen om te wennen aan bepaalde functionaliteiten bij een gebrek aan in-app user training of onboarding. Ook verwacht ik dat een onboarding waarbij alle functionaliteiten aan bod komen een negatieve impact geeft in vergelijking met in-app user training per functionaliteit wanneer de gebruiker hiermee in contact komt.

%---------- Verwachte conclusies ----------------------------------------------
\section{Verwachte conclusies}
\label{sec:verwachte_conclusies}

Het nut van een goede onboarding zal bewezen worden. Alsook mag de onboarding enkel betrekking hebben op de meest essentiële functionaliteit.

In-app user training zal ervoor zorgen dat de eindgebruiker het gevoel heeft de functionaliteit onder de knie te hebben. Daardoor zal de gebruiker minder vlug afstand nemen van de applicatie. Ook zal het eenvoudiger zijn voor gebruikers met een beperkte technische achtergrond om gebruik te maken van de applicatie.

